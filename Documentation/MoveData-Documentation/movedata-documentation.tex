% ==================
% Standalone support
% ==================


\makeatletter
\ifdefined\preloaded
\else
	\newcommand{\preloaded}{not set!}
	\ifx\documentclass\@twoclasseserror % after \documentclass
		\renewcommand{\preloaded}{true}
	\else
		\documentclass[12pt]{report}

% Packages
% ========
\usepackage[letterpaper, portrait, margin=1in]{geometry}
\usepackage{graphicx}			% So we can load figures with "." in their filename
\usepackage{float}				% So we can use "[H]"
\usepackage[dvipsnames]{xcolor}	% For custom colors
\usepackage{xparse}				% For \DeclareDocumentCommand
\usepackage{caption}			% For \captionof command
\usepackage{mathtools}			% For \DeclarePairedDelimeter
\usepackage{ifthen}				% For \ifthenelse{}{}{}
\usepackage{soul}				% For underlining with \ul
\setuldepth{x}					%	""
\usepackage{multicol}			% For multiple columns via \begin{multicols}{2}
\usepackage[most]{tcolorbox}	% For the tl; dr environment
\usepackage{array}				% For extended column definitions
\usepackage{tabularray}			% For \begin{longtblr} tables that span multiple columns and pages
\usepackage{amssymb}			% For $\checkmark$
\usepackage{etoc}				% For \localtableofcontents


% Inserting code into LaTeX: \begin{lstlisting} ...
\usepackage{listings}
\definecolor{dkgreen}{rgb}{0,0.6,0}
\definecolor{gray}{rgb}{0.5,0.5,0.5}
\definecolor{mauve}{rgb}{0.58,0,0.82}

\lstset{frame=tb,
  language=C++,
  aboveskip=3mm,
  belowskip=3mm,
  showstringspaces=false,
  columns=flexible,
  basicstyle={\small\ttfamily},
  numbers=none,
  numberstyle=\tiny\color{gray},
  keywordstyle=\color{blue},
  commentstyle=\color{dkgreen},
  stringstyle=\color{mauve},
  breaklines=true,
  breakatwhitespace=true,
  tabsize=3
}

% Inline code
\definecolor{darkpink}{rgb}{0.5, 0.0, 0.5}
\newcommand{\code}[1]{\texttt{\color{darkpink}#1}}

% Bibliography
% ============
\usepackage[american]{babel}
\usepackage{csquotes}
\usepackage[style=ieee, backend=biber]{biblatex}
\usepackage{hyperref}
\hypersetup{colorlinks=true, allcolors=black, urlcolor=blue}
\addbibresource{../sources.bib}

% etoc setup
% ==========

\etocsetstyle{section}{}{}{\etocsavedchaptertocline{\numberline{}\etocname}{\etocpage}}{}
\etocsetstyle{subsection}{}{}{\etocsavedsectiontocline{\numberline{}\etocname}{\etocpage}}{}
\etocsetstyle{subsubsection}{}{}{\etocsavedsubsectiontocline{\numberline{}\etocname}{\etocpage}}{}
\etocsetstyle{paragraph}{}{}{\etocsavedsubsubsectiontocline{\numberline{}\etocname}{\etocpage}}{}
\etocsetstyle{subparagraph}{}{}{\etocsavedparagraphtocline{\numberline{}\etocname}{\etocpage}}{}


% Simple custom commands
% ======================
\makeatletter
\def\maxwidth#1{\ifdim\Gin@nat@width>#1 #1\else\Gin@nat@width\fi} 
\makeatother

\def\todo#1{\selectfont{\color{red}\texttt{\textbf{TODO:} #1}}}

% Sections 'n' such
% =================

\definecolor{chaptColor}{RGB}{0, 83, 161}

\def\chapt#1{
%
	% Default chapter behavior
	\begingroup\color{chaptColor}
	\chapter{#1}
	\endgroup
	
	% Label
	\label{chp:#1}
}

\definecolor{sectColor}{rgb}{0, 0.5, 0.0}
\def\sect#1{\textcolor{sectColor}{\section{#1}}}

\definecolor{subsectColor}{rgb}{0, 0.5, 0.5}
\def\subsect#1{\textcolor{subsectColor}{\subsection{#1}}\noindent}

\definecolor{subsubsectColor}{rgb}{0.747, 0.458, 0}
\def\subsubsect#1{\textcolor{subsubsectColor}{\subsubsection{#1}}}

% TL; DR section
% ==============
\newenvironment{tldr}{\begin{tcolorbox}[colback=gray!20!white,colframe=blue!75!black,title=TL; DR]}{\end{tcolorbox}\vspace*{12pt}}

% Custom math commands
% ====================
\DeclarePairedDelimiter\ceil{\lceil}{\rceil}
\DeclarePairedDelimiter\floor{\lfloor}{\rfloor}

% ======================
% \graphic{filename=...}
% ======================
% Keyword arguments
\makeatletter
\define@key{graphicKeys}{filename}{\def\graphic@filename{#1}}
\define@key{graphicKeys}{scale}{\def\graphic@scale{#1}}
\define@key{graphicKeys}{width}{\def\graphic@width{#1}}
\define@key{graphicKeys}{caption}{\def\graphic@caption{#1}}
\define@key{graphicKeys}{captionType}{\def\graphic@captionType{#1}}
\define@key{graphicKeys}{label}{\def\graphic@label{#1}}

% Kwargs
\DeclareDocumentCommand{\graphic}{m}{

	\begingroup
	
	% Set default kwargs
	\setkeys{graphicKeys}{filename={0}, #1}
	\setkeys{graphicKeys}{scale={0}, #1}
	\setkeys{graphicKeys}{width={0}, #1}
	\setkeys{graphicKeys}{caption={}, #1}
	\setkeys{graphicKeys}{captionType={figure}, #1}
	\setkeys{graphicKeys}{label={}, #1}
	
	% Assign width
	\let\graphicWidth\relax % let \mytmplen to \relax
	\newlength{\graphicWidth}
	\setlength{\graphicWidth}{\columnwidth}
	
	\if \graphic@scale 0
		
		\if \graphic@width 0
			
			\setlength{\graphicWidth}{0.5\columnwidth}
		
		\else
	
			\setlength{\graphicWidth}{\graphic@width}
			
		\fi

	\else
		
		\setlength{\graphicWidth}{\columnwidth * \graphic@scale}

	\fi	
	
	% Do figure
	\begin{minipage}{\columnwidth}
		\vspace*{12pt}
		\begin{center}
		
			\includegraphics[width = \graphicWidth]{\graphic@filename}
			
			\ifthenelse{\equal{\graphic@caption}{}}{}{
				\captionof{\graphic@captionType}{\graphic@caption}
			}
			
			\ifthenelse{\equal{\graphic@label}{}}{
				\label{fig: \graphic@filename}
			}{
				\label{\graphic@label}
			}
		\end{center}
		\vspace*{12pt}
	\end{minipage}
	
	\endgroup
}
\makeatother

% ==============
% Base Pair Page
% ==============
% Keyword arguments
\makeatletter
\define@key{bpPageKeys}{baseFilename}{\def\bpPage@baseFilename{#1}}
\define@key{bpPageKeys}{scale}{\def\bpPage@scale{#1}}
\define@key{bpPageKeys}{width}{\def\bpPage@width{#1}}
\define@key{bpPageKeys}{caption}{\def\bpPage@caption{#1}}
\define@key{bpPageKeys}{captionType}{\def\bpPage@captionType{#1}}
\define@key{bpPageKeys}{label}{\def\bpPage@label{#1}}

% Kwargs
\DeclareDocumentCommand{\BPPage}{m}{

	\begingroup
	
	% Set default kwargs
	\setkeys{bpPageKeys}{baseFilename={0}, #1}
	\setkeys{bpPageKeys}{caption={}, #1}
	\setkeys{bpPageKeys}{label={}, #1}
	
	\newpage

	\begin{center}
		\ul{\mbox{\bpPage@baseFilename{}: 1 Base Pair [min]}}
	\end{center}
	\vspace*{-36pt}
	\begin{minipage}[b]{0.5\textwidth}
		\graphic{filename=\bpPage@baseFilename-overview-1-table, width=\textwidth}
	\end{minipage}
	\begin{minipage}[b]{0.5\textwidth}
		\graphic{filename=\bpPage@baseFilename-overview-1-plot, width=\textwidth}
	\end{minipage}

	\begin{center}
		\ul{\mbox{\bpPage@baseFilename{}: 50 Base Pairs [average]}}
	\end{center}
	\vspace*{-36pt}
	\begin{minipage}[b]{0.5\textwidth}
		\graphic{filename=\bpPage@baseFilename-overview-50-table, width=\textwidth}
	\end{minipage}
	\begin{minipage}[b]{0.5\textwidth}
		\graphic{filename=\bpPage@baseFilename-overview-50-plot, width=\textwidth}
	\end{minipage}
	
	\begin{center}
		\ul{\mbox{\bpPage@baseFilename{}: 100 Base Pairs [max]}}
	\end{center}
	\vspace*{-36pt}
	\begin{minipage}[b]{0.5\textwidth}
		\graphic{filename=\bpPage@baseFilename-overview-100-table, width=\textwidth}
	\end{minipage}
	\begin{minipage}[b]{0.5\textwidth}
		\graphic{filename=\bpPage@baseFilename-overview-100-plot, width=\textwidth}
	\end{minipage}

	\vspace*{-24pt}
	\captionof{figure}{\bpPage@caption}\label{\bpPage@label}	
	
	\endgroup
}
\makeatother



\begin{document}
  		\begin{document}
		\renewcommand{\preloaded}{false}
	\fi
\fi
\makeatother
\trysetmain{}

% Additional usage: \ifthenelse{\equal{\preloaded}{true}}{ ... }{ ... }
% Note: Cannot do \trysetmain{} here because the \currfilename will always be "standalone.tex"

\begin{tldr}
	\begin{itemize}
		\item{Moves have Categories, which can be seen in Table~\ref{move-cat-table}.}
		\item{Similarly, Moves can make contact (think Spiky Shield), which can be seen in Table~\ref{move-contact-table}.}
	\end{itemize}
\end{tldr}

\localtoc

%====================================================

\sect{What Is \code{MoveData}?}

\code{MoveData} is an asset that contains static data on a Move. Specifically, it contains information on:\\

\newcommand{\bitem}[2]{\item{\textbf{#1}. #2}}

\begin{itemize}
	\bitem{Base power}{A measure of the relative power of a Move. Starter Moves may only have 10--30 base power, whereas the most powerful moves may tip the scales at 100. The majority of Moves should have low base powers---we don't want the OHKOs of Pokemon, as this isn't a turn-based game.}
	\bitem{Base cooldown}{The cooldown in seconds. This gets modified by the \code{CombatStatsComponent}'s Haste.}
	\bitem{Category}{A representation of the kind of Move this is. The \code{MoveCategory}s are:
	\begin{longtblr}[
		caption = {All \code{MoveCategory}s currently implemented.},
		label = {move-cat-table},
	]{
		colspec={Q[l, wd=0.17\textwidth] Q[l, wd=0.6\textwidth]},
		hlines,
		row{1} = {font=\bfseries, fg=ICBlue},
		colsep = 0.005\textwidth,
	}
	Name 	& Description\\
	\code{None}	& Default. Don't use this. If you see it, it's an error.\\
	\code{PhysicalDamage}	& If damaging, the Move relies on the user's physical abilities, like strength.\\
	\code{SpecialDamage} & If damaging, the Move relies on the user's non-physical abilities, like psychic or spiritual ability.\\
	\code{PhysicalHealing} & A move that restores health (self, allies, enemies, etc.) by physical means.  For example, an injection, food, a bandage, or acupuncture.\\
	\code{SpecialHealing} & A move that restores health (self, allies, enemies, etc.) non-physically.  For example, mental healing, emotional support, spiritual peace, or the reversing of time to heal wounds.\\
	\code{Summoning} & A move that summons an object or creature.\\
	\code{Utility} & Does no damage to opponent or self. May still attach effects. Pokemon example: Thunder Wave or Pay Day (but without damage).
	\end{longtblr}
	}
	\bitem{Contact}{Determines what kind of contact (if any) occurs when the Move hits a target. The \code{MoveCategory}s are:
	\begin{longtblr}[
		caption = {All \code{MoveContact}s currently implemented.},
		label = {move-contact-table},
	]{
		colspec={Q[l, wd=0.17\textwidth] Q[l, wd=0.6\textwidth]},
		hlines,
		row{1} = {font=\bfseries, fg=ICBlue},
		colsep = 0.005\textwidth,
	}
	Name 	& Description\\
	\code{None}	& No contact (e.g., Summoning or damaging from afar).\\
	\code{PhysicalContact} & The damage or status requires the user to make physical contact with the target. Example: punching something.\\
	\code{SpecialContact} & The damage or status requires the user to make non-physical contact with the target. Examples: direct psychic connection, a sustained electrical current from the user to the target, or a ``drain life'' effect.
	\end{longtblr}
	I'll concede that \code{SpecialContact} probably isn't super useful to track, but you never know. It's easier to put it in now rather than later.
	}
	\bitem{Display name}{If the field \code{DisplayName} is blank, the asset's name will be used. May be useful for special characters.}
	\bitem{Effects to inflict}{A \code{TMap} of \code{EffectComponent}s and thier percentages of being applied on a successful Move hit. Pokemon example: a Move having 30\% chance to Burn would have its \code{TPair} be \code{Burn} and \code{30}.}
	\bitem{Randomness}{A range of values to multiply to calculate the final damage. Normally, any Move's damage is multiplied by a number between 0.85 and 1 (range taken, of course, from Pokemon). This variance makes it a little more exciting.}
	\bitem{Supporting text}{\code{SupportingText} includes dev notes that doesn't appear in-game, a proper description, and flavor text (e.g., for hovers).}
	\bitem{Move type(s)}{The incredibly vast majority of moves should only associate with one Type. It gets far too complicated and unpredictably unfun to have multi-type attacks.}
\end{itemize}

%====================================================

\sect{Existing \code{MoveData}}

\begin{longtblr}[
		caption = {All \code{MoveContact}s currently implemented.},
		label = {move-contact-table},
	]{
		colspec={Q[l, wd=0.17\textwidth] Q[l, wd=0.10\textwidth] Q[l, wd=0.08\textwidth] Q[l, wd=0.17\textwidth] Q[l, wd=0.17\textwidth]},
		hlines,
		row{1} = {font=\bfseries, fg=ICBlue},
		colsep = 0.005\textwidth,
	}
	Name & Type & Base Power & Effects & Notes \\
	Blip & Electric & 20 &  & \\
\end{longtblr}

%====================================================

\sect{Damage Calculation Formula}

The formulas for damage and healing are inspired by Pokemon. Like it or not, that's the gold standard and what most players will expect when judging base powers, stat values, etc.

The formulas are implemented in \code{CombatStatsComponent.cpp}:
\begin{align*}
	\text{Damage} &= \left[\left(0.4\times \text{Level} + 2\right) \times \frac{\text{Base Power}}{50} \times \frac{\text{Attack}}{\text{Defense}} + 2\right]\\
		& \qquad \times \text{Crit Multiplier} \times \text{Random Fluctuation} \times \text{STAB Multiplier} \times \text{Type Advantage}\\
		& \qquad \times 3^{\floor{\frac{\text{Level}}{10}}}
\end{align*}
where:
\begin{itemize}
	\item{Level is the attacker's Level}
	\item{Base Power is determined by the \code{MoveData}}
	\item{Attack and Defense are the appropriate stat values}
	\item{Crit Multiplier is 1.5$\times$ if a critical hit occurred; 1.0$\times$ otherwise
		\begin{itemize}
			\item{As outlined in the Critical Hit section of the Stats documentation, if a crit does occur, this reverts Defense to its un-buffed value (including everything, e.g., Mutations).}
			\item{If the Critical Hit stat is above 100\%, the critical damage increases accordingly. For example, if Crit is 120\%, the damage bonus is $1.5 + 0.2 = 1.7\times$.}
			\item{This can be a dangerous game mechanic and will require a lot of balancing. However, it also opens the strategies in an otherwise traditional stat scheme.}
		\end{itemize}
	}
	\item{Random Fluctuation is determined by the ``randomness'' of the \code{MoveData} (in particular, the field \code{MoveData.RandomRange}). Nominally, this is 0.85--1.}
	\item{\ul{S}ame \ul{T}ype \ul{A}ttack \ul{B}onus (STAB) rewards Monsters for using Moves that share their Types (e.g., a Fire-type Monster using a Fire-type Move). Nominally, this is 1.5$\times$, but can be more or less (e.g., for Typeless Monsters, it's 2$\times$).}
	\item{Type Advantage is outlined in the Types documentation. Normally, it ranges anywhere from $-1$ (Electric actually heals Electric) to $0.25\times$ (double resisted) to $4\times$ (doubly weak). It follows established Pokemon conventions.}
\end{itemize}
Similarly, the healing formula is:
\begin{align*}
	\text{Healing} &= \left[\left(0.4\times \text{Level} + 2\right) \times \frac{\text{Base Power}}{50} \times \frac{\text{Attack}}{\text{Reference Stat}} + 2\right]\\
		& \qquad \times \text{Crit Multiplier} \times \text{Random Fluctuation} \times \text{STAB Multiplier}\\
		& \qquad \times 3^{\floor{\frac{\text{Level}}{10}}}
\end{align*}
Here, Reference~Value is a dummy \code{StandardStat} (like Physical Defense, for example) with:
\begin{itemize}
	\item{Base Stat set to 100}
	\item{Base Pairs set to 100}
	\item{Level set to the defending Monster's Level}
\end{itemize}

%====================================================

\sect{Damage Calculation Case Studies}

Here, we analyze an attacking Monster versus a defending Monster. For simplicity:\\
\begin{itemize}
	\item{Both Monsters had perfect Base Pairs in all stats}
	\item{Both Monsters Levels are equal (so a Level~52 attacker goes against a Level~52 defender)}
	\item{Crit chance was 0\%}
	\item{Random fluctuations were 1.0}
	\item{STAB and type advantage multipliers were 1.0}
\end{itemize}

\begin{multicols}{2}

\graphic{filename=../MoveData-Documentation/base-attack-variation,
	caption={Damage calculation at various Levels and Base Attack stats. Defender had 100~Base Defense and the Move had a base power of 50.},
	width=\columnwidth
	}
	
\graphic{filename=../MoveData-Documentation/base-defense-variation,
	caption={Damage calculation at various Levels and Base Defense stats. Attacker had 100~Base Attack and the Move had a base power of 50. Not surprisingly, this was the inverse of the previous plot.},
	width=\columnwidth
	}

\end{multicols}

\noindent Okay, so at Level 100, they're dealing, like, 5~million damage. More interestingly, let's vary the base power, leaving the base stats alone:
\graphic{filename=../MoveData-Documentation/base-power-variation,
	caption={Raw damage calculation at various Levels and base powers. Both attacker and defender are at the same level. Remember: we want a ``good'' endgame Move to have 50 base power.},
	width=\columnwidth
	}
\graphic{filename=../MoveData-Documentation/base-power-variation-normed,
	caption={This time, damage is represented as the percentage of the defender's HP. Both attacker and defender are at the same level. Initially, we have fast games, but as the player gets into it, the games slow down and the player has to think harder about choices like Type matchups. Remember: we want a ``good'' endgame Move to have 50 base power.},
	width=\columnwidth
	}
	
\newpage

\noindent Lastly, let's look at an underleveled team in the endgame:

\graphic{filename=../MoveData-Documentation/endgame-damage-plot,
	caption={Attacker at various Levels versus a Lv.~100 defender. I think it adds to the sense of accomplishment once a Monster reaches Lv.~100 and you \textit{really} start seeing those big damage numbers. It also makes for a way for grind to make up for lack of skill.},
	width=\columnwidth
	}

%====================================================

\postamble{}