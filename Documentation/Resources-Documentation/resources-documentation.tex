% ==================
% Standalone support
% ==================


\makeatletter
\ifdefined\preloaded
\else
	\newcommand{\preloaded}{not set!}
	\ifx\documentclass\@twoclasseserror % after \documentclass
		\renewcommand{\preloaded}{true}
	\else
		\documentclass[12pt]{report}

% Packages
% ========
\usepackage[letterpaper, portrait, margin=1in]{geometry}
\usepackage{graphicx}			% So we can load figures with "." in their filename
\usepackage{float}				% So we can use "[H]"
\usepackage[dvipsnames]{xcolor}	% For custom colors
\usepackage{xparse}				% For \DeclareDocumentCommand
\usepackage{caption}			% For \captionof command
\usepackage{mathtools}			% For \DeclarePairedDelimeter
\usepackage{ifthen}				% For \ifthenelse{}{}{}
\usepackage{soul}				% For underlining with \ul
\setuldepth{x}					%	""
\usepackage{multicol}			% For multiple columns via \begin{multicols}{2}
\usepackage[most]{tcolorbox}	% For the tl; dr environment
\usepackage{array}				% For extended column definitions
\usepackage{tabularray}			% For \begin{longtblr} tables that span multiple columns and pages
\usepackage{amssymb}			% For $\checkmark$
\usepackage{etoc}				% For \localtableofcontents


% Inserting code into LaTeX: \begin{lstlisting} ...
\usepackage{listings}
\definecolor{dkgreen}{rgb}{0,0.6,0}
\definecolor{gray}{rgb}{0.5,0.5,0.5}
\definecolor{mauve}{rgb}{0.58,0,0.82}

\lstset{frame=tb,
  language=C++,
  aboveskip=3mm,
  belowskip=3mm,
  showstringspaces=false,
  columns=flexible,
  basicstyle={\small\ttfamily},
  numbers=none,
  numberstyle=\tiny\color{gray},
  keywordstyle=\color{blue},
  commentstyle=\color{dkgreen},
  stringstyle=\color{mauve},
  breaklines=true,
  breakatwhitespace=true,
  tabsize=3
}

% Inline code
\definecolor{darkpink}{rgb}{0.5, 0.0, 0.5}
\newcommand{\code}[1]{\texttt{\color{darkpink}#1}}

% Bibliography
% ============
\usepackage[american]{babel}
\usepackage{csquotes}
\usepackage[style=ieee, backend=biber]{biblatex}
\usepackage{hyperref}
\hypersetup{colorlinks=true, allcolors=black, urlcolor=blue}
\addbibresource{../sources.bib}

% etoc setup
% ==========

\etocsetstyle{section}{}{}{\etocsavedchaptertocline{\numberline{}\etocname}{\etocpage}}{}
\etocsetstyle{subsection}{}{}{\etocsavedsectiontocline{\numberline{}\etocname}{\etocpage}}{}
\etocsetstyle{subsubsection}{}{}{\etocsavedsubsectiontocline{\numberline{}\etocname}{\etocpage}}{}
\etocsetstyle{paragraph}{}{}{\etocsavedsubsubsectiontocline{\numberline{}\etocname}{\etocpage}}{}
\etocsetstyle{subparagraph}{}{}{\etocsavedparagraphtocline{\numberline{}\etocname}{\etocpage}}{}


% Simple custom commands
% ======================
\makeatletter
\def\maxwidth#1{\ifdim\Gin@nat@width>#1 #1\else\Gin@nat@width\fi} 
\makeatother

\def\todo#1{\selectfont{\color{red}\texttt{\textbf{TODO:} #1}}}

% Sections 'n' such
% =================

\definecolor{chaptColor}{RGB}{0, 83, 161}

\def\chapt#1{
%
	% Default chapter behavior
	\begingroup\color{chaptColor}
	\chapter{#1}
	\endgroup
	
	% Label
	\label{chp:#1}
}

\definecolor{sectColor}{rgb}{0, 0.5, 0.0}
\def\sect#1{\textcolor{sectColor}{\section{#1}}}

\definecolor{subsectColor}{rgb}{0, 0.5, 0.5}
\def\subsect#1{\textcolor{subsectColor}{\subsection{#1}}\noindent}

\definecolor{subsubsectColor}{rgb}{0.747, 0.458, 0}
\def\subsubsect#1{\textcolor{subsubsectColor}{\subsubsection{#1}}}

% TL; DR section
% ==============
\newenvironment{tldr}{\begin{tcolorbox}[colback=gray!20!white,colframe=blue!75!black,title=TL; DR]}{\end{tcolorbox}\vspace*{12pt}}

% Custom math commands
% ====================
\DeclarePairedDelimiter\ceil{\lceil}{\rceil}
\DeclarePairedDelimiter\floor{\lfloor}{\rfloor}

% ======================
% \graphic{filename=...}
% ======================
% Keyword arguments
\makeatletter
\define@key{graphicKeys}{filename}{\def\graphic@filename{#1}}
\define@key{graphicKeys}{scale}{\def\graphic@scale{#1}}
\define@key{graphicKeys}{width}{\def\graphic@width{#1}}
\define@key{graphicKeys}{caption}{\def\graphic@caption{#1}}
\define@key{graphicKeys}{captionType}{\def\graphic@captionType{#1}}
\define@key{graphicKeys}{label}{\def\graphic@label{#1}}

% Kwargs
\DeclareDocumentCommand{\graphic}{m}{

	\begingroup
	
	% Set default kwargs
	\setkeys{graphicKeys}{filename={0}, #1}
	\setkeys{graphicKeys}{scale={0}, #1}
	\setkeys{graphicKeys}{width={0}, #1}
	\setkeys{graphicKeys}{caption={}, #1}
	\setkeys{graphicKeys}{captionType={figure}, #1}
	\setkeys{graphicKeys}{label={}, #1}
	
	% Assign width
	\let\graphicWidth\relax % let \mytmplen to \relax
	\newlength{\graphicWidth}
	\setlength{\graphicWidth}{\columnwidth}
	
	\if \graphic@scale 0
		
		\if \graphic@width 0
			
			\setlength{\graphicWidth}{0.5\columnwidth}
		
		\else
	
			\setlength{\graphicWidth}{\graphic@width}
			
		\fi

	\else
		
		\setlength{\graphicWidth}{\columnwidth * \graphic@scale}

	\fi	
	
	% Do figure
	\begin{minipage}{\columnwidth}
		\vspace*{12pt}
		\begin{center}
		
			\includegraphics[width = \graphicWidth]{\graphic@filename}
			
			\ifthenelse{\equal{\graphic@caption}{}}{}{
				\captionof{\graphic@captionType}{\graphic@caption}
			}
			
			\ifthenelse{\equal{\graphic@label}{}}{
				\label{fig: \graphic@filename}
			}{
				\label{\graphic@label}
			}
		\end{center}
		\vspace*{12pt}
	\end{minipage}
	
	\endgroup
}
\makeatother

% ==============
% Base Pair Page
% ==============
% Keyword arguments
\makeatletter
\define@key{bpPageKeys}{baseFilename}{\def\bpPage@baseFilename{#1}}
\define@key{bpPageKeys}{scale}{\def\bpPage@scale{#1}}
\define@key{bpPageKeys}{width}{\def\bpPage@width{#1}}
\define@key{bpPageKeys}{caption}{\def\bpPage@caption{#1}}
\define@key{bpPageKeys}{captionType}{\def\bpPage@captionType{#1}}
\define@key{bpPageKeys}{label}{\def\bpPage@label{#1}}

% Kwargs
\DeclareDocumentCommand{\BPPage}{m}{

	\begingroup
	
	% Set default kwargs
	\setkeys{bpPageKeys}{baseFilename={0}, #1}
	\setkeys{bpPageKeys}{caption={}, #1}
	\setkeys{bpPageKeys}{label={}, #1}
	
	\newpage

	\begin{center}
		\ul{\mbox{\bpPage@baseFilename{}: 1 Base Pair [min]}}
	\end{center}
	\vspace*{-36pt}
	\begin{minipage}[b]{0.5\textwidth}
		\graphic{filename=\bpPage@baseFilename-overview-1-table, width=\textwidth}
	\end{minipage}
	\begin{minipage}[b]{0.5\textwidth}
		\graphic{filename=\bpPage@baseFilename-overview-1-plot, width=\textwidth}
	\end{minipage}

	\begin{center}
		\ul{\mbox{\bpPage@baseFilename{}: 50 Base Pairs [average]}}
	\end{center}
	\vspace*{-36pt}
	\begin{minipage}[b]{0.5\textwidth}
		\graphic{filename=\bpPage@baseFilename-overview-50-table, width=\textwidth}
	\end{minipage}
	\begin{minipage}[b]{0.5\textwidth}
		\graphic{filename=\bpPage@baseFilename-overview-50-plot, width=\textwidth}
	\end{minipage}
	
	\begin{center}
		\ul{\mbox{\bpPage@baseFilename{}: 100 Base Pairs [max]}}
	\end{center}
	\vspace*{-36pt}
	\begin{minipage}[b]{0.5\textwidth}
		\graphic{filename=\bpPage@baseFilename-overview-100-table, width=\textwidth}
	\end{minipage}
	\begin{minipage}[b]{0.5\textwidth}
		\graphic{filename=\bpPage@baseFilename-overview-100-plot, width=\textwidth}
	\end{minipage}

	\vspace*{-24pt}
	\captionof{figure}{\bpPage@caption}\label{\bpPage@label}	
	
	\endgroup
}
\makeatother



\begin{document}
  		\begin{document}
		\renewcommand{\preloaded}{false}
	\fi
\fi
\makeatother
\trysetmain{}

% Additional usage: \ifthenelse{\equal{\preloaded}{true}}{ ... }{ ... }
% Note: Cannot do \trysetmain{} here because the \currfilename will always be "standalone.tex"

\begin{tldr}
	The resources are:
	\begin{itemize}
		\item{Monsters (rarities and such)}
		\item{Stats (see Stats document)}
		\item{Exp}
		\item{Affinity Points (basically DIY Pok\'{e}mon types)}
		\item{Player talent points (standard RPG stuff)}
		\item{Moves (learned via level-up or scrolls)}
		\item{Items (e.g., stat enhancement)}
		\item{``Gold'' (as a reward; to buy items or attempts)}
	\end{itemize}
	
	There are generally 4 ways to obtain these resources:
	\begin{itemize}
		\item{Grinding in the world}
		\item{Buy with gold (\textbf{no in-app purchases!})}
		\item{Gifts from NPCs}
		\item{The Colosseum}
	\end{itemize}
\end{tldr}

\sect{Monsters}
Since the game is about collecting Monsters, Monsters, of course, are a restricted resource that drives the player to progress the story and explore new areas. There are a few ways to obtain Monsters:
\begin{itemize}
	\item{\textbf{NPC gifts.} The first Monster that the player obtains will likely be a gift (although it certainly could be purchasable with the Player's starting money).}
	\item{\textbf{Finding in the wild.} Some Monsters appear uniquely in areas and others are found throughout the world. The probability of finding a Monster in the wild greatly varies. Some are even a sub-1\% chance.}
	\item{\textbf{NPC trades.} Some NPCs will trade one Monster for another. While these traded Monsters should not be unique, trading should allow for the Player to get Monsters that are not indigenous to the surrounding areas.}
	\item{\textbf{NPC purchases.} This is \textbf{NOT} an in-game purchase for real money! Rather, there will be a mechanic for the Player to spend ``gold'' (see section below) in a black-market--esque area to buy Monsters. The price will be steep, and so the Player will not be able to easily obtain these Monsters.}
\end{itemize}

\sect{Stats}
Monsters use Stats in combat to determine a win or a loss. The Stats chapter contains much more detail, but here it is discussed as a resource.
\begin{itemize}
	\item{Stats increase as the Monster increases in level.}
	\item{Stats ``jump'' every 10 levels. This is soft gatekeeping for areas that the Player should not yet visit.}
\end{itemize}

\sect{Exp and Levels}
A Monster's Stats are directly related to its level. The Monster's level is the same as its cumulative exp:

\begin{equation}
	\texttt{level} = \left\lfloor \texttt{cumulative exp}^{1/3} \right\rfloor
\end{equation}

Exp can only be gained by doing battle. Each Monster yields a different amount of exp based on its level and its opponent's level.

\sect{Affinity Points}
Affinity points are a measure of how much of a particular element a Monster is. As an example, take Flygon from the Pok\'{e}mon series \cite{pkmn-website}. 

\graphic{filename=flygon,
	label={fig: flygon},
	width=0.5\textwidth,
	caption={Flygon from the Pok\'{e}mon series. Flygon's type is Ground/Dragon, but it could easily also be Bug or Flying.}
}

Flygon's actual type is Ground/Dragon. However, when Flygon was being created, the developers could have easily chosen \textit{any two} of the following types: Bug, Flying, Ground, Dragon. 

If Flygon were a Monster instead of a Pok\'{e}mon, it would have a built-in Earth Affinity, with options for Air, Nature, or Magic affinities.

Too many Elements are bad for the game. Pok\'{e}mon doesn't do this because it would be too confusing to calculate advantages and disadvantages for a quad-type Pok\'{e}mon, and the game would lose some of its enjoyment. For this reason, Monsters are normally capped at two Affinities.\footnote{Bosses may have any number of Affinities for novelty's sake, and some Monsters may only have 1 Affinity.}

Figure~\ref{fig: affinities-example} shows Void Rhino's Affinities. Void Rhino can be pure Void, Void/Ice, or Void Earth. However, it cannot be Ice/Earth or similar. Looking at Figure~\ref{fig: affinities-example}, the checkbox on Void Rhino's Void Affinity is ``locked'', meaning that all Void Rhinos \textbf{must} have at least one Affinity point in Void (it's in their name, after all).

\graphic{filename=affinities-example,
	width=\textwidth,
	caption={Void Rhino's Affinities (Void, Ice, and Earth). Note that the Void Element is locked, meaning that one of Void Rhino's Affinities must always be Void.}
}

There is also an option for the degree of Affinity. In the aforementioned example, this Void Rhino has 3 Affinity points in Void and 2 Affinity points in Ice. These grant the following bonuses:

\begin{itemize}
	\item{1 Affinity point lets the Monster use STAB\footnote{\ul{S}ame \ul{T}ype \ul{A}ttack \ul{B}onus.} for that Element and all of the weaknesses and resistances that come with it.}
	\item{2 Affinity points allows the Monster to use ``Lesser Scrolls''\footnote{Name pending.} to learn Moves of that Element.}
	\item{3 Affinity points allows the Monster to use ``Greater Scrolls'' to learn powerful Moves of that element.}
\end{itemize}

Currently, a Monster may only have a total of 6 Affinity points that they receive every so often. \todo{How often, exactly? Every x levels, or after certain story progressions, like defeating a boss?}

So that there is some degree of predictability in combat, a Monster's color scheme will slightly change depending on their Affinities. \todo{Do this!}

Finally, Affinity points may be reset at any time for free. This allows the Player to experiment and customize without fear of making a big permanent mistake.

\sect{Talent Points}
Like many good RPGs, the Player has additional customization through talent points. The mechanics of this need not be dissected, as they are very standard. \todo{How often does the Player recieve talent points? What do talent trees look like? Are they WoW-WotLK--style branches?}	

\sect{Moves}
There are two types of ways to acquire Moves:
\begin{itemize}
	\item{\textbf{Moves learnable by level-up.} Like in Pok\'{e}mon, Moves learned via level-up are not rewarded in set level increments, but vary from Monster to Monster.}
	\item{\textbf{Scrolls.} Scrolls are a TM-like mechanic, but a TM never made sense to me (you insert a disc into a Pok\'{e}mon or something?). Scrolls probably don't make sense either, and should probably be renamed to something more sciencey.}
	\item{\textbf{Please}, no mechanics like move tutors or breeding. Moves exist to define and differentiate Monsters. If there are other avenues to learn Moves, combat may become homogenized and some Monsters may start to strictly outclass others.}
\end{itemize}

\sect{Items}
Items have various uses, but there will be no usable in-battle items (such as Pok\'{e}mon's Full Restore). Some examples are:

\begin{itemize}
	\item{Items to increase DNA Base Pairs up to:
		\begin{itemize}
			\item{60 (common)}
			\item{70 (uncommon)}
			\item{80 (rare)}
			\item{90 (extremely rare; only as a reward from the Colosseum)}
		\end{itemize}
	}
	\item{Items that, in conjunction with gold, re-roll:
		\begin{itemize}
			\item{Genetic Mutation}
			\item{Aura}
			\item{Color variation}
		\end{itemize}
	}
	\item{Scrolls that teach Moves}
\end{itemize}
While no held items are planned for battles, they may be in the future.

\sect{``Gold''}
``Gold'' is a stand-in name for money. Like exp, gold may be acquired from battles, but may also be acquired in other ways. For example, the Player may find gold in the world or may be gifted gold from NPCs. Gold may be used for:
\begin{itemize}
	\item{Purchasing Monsters}
	\item{Purchasing items (such as scrolls; see ``Items'' section)}
	\item{Re-rolling (e.g., DNA Base Pairs). See ``Items'' section. This should be very expensive, and should follow Figure~\ref{fig: reroll-ivs}, where the more Base Pairs you have locked, the more expensive it is.}
\end{itemize}

\graphic{filename=reroll-ivs,
	width=\textwidth,
	caption={Screenshot from a Pok\'{e}mon mobile rip-off game called ``Monster Carnival'' (or ``Pixelmon Town'', depending on if they got shut down or not). While the game was an obvious attempt to get people to part with their money, they did have good, entertaining mechanics for upgrading Pok\'{e}mon that the main series did not. In this screenshot, IVs may be ``locked'' so that when the Player re-rolls, the locked values do not change. However, the more locked IVs there are, the more expensive it becomes to re-roll.}
}

\sect{``Colosseum''}
``Colosseum'' is a stand-in name for a mode similar to Magic: The Gathering draft or Hearthstone Arena. The mode opens up mid-game \todo{level 35?}. It is infinitely repeatable: it does not cost gold and has no daily cooldown. It is just for fun and to acquire Monsters if the Player is skilled or lucky enough.

It serves to enhance a Player's team if they need it or to satisfy min/max players. Its purpose is inspired by the way the Pok\'{e}mon daycare is usually available to mid-game players. Players may also acquire end-game Monsters before they would normally be able to.

\subsect{Setup}
The Player starts with an empty team. The Player is then offered three randomized Monsters. They can only choose one to be on their team. The offer is repeated six times (or however many make up a party) until the Player has a full team. The pool of Monsters is not very restrictive, and the Player may get to use some Colosseum-exclusive Monsters.

The Player then competes until they lose 3 \todo{3? 1?} times.

Smogon has (used to have?) a random battle mode that was quite fun (even when I got a level 100 Unown). This is definitely in the spirit of that.

\subsect{Rewards}
Depending on how many times the Player wins, they get rewards. Some examples of rewards:
\begin{itemize}
	\item{Win 3+
		\begin{itemize}
			\item{Keep 1 drafted Monster}
			\item{Win gold}
		\end{itemize}
	}
	\item{Win 5+
		\begin{itemize}
			\item{Keep 2 drafted Monster}
			\item{Win more gold}
		\end{itemize}
	}
	\item{...etc.}
	\item{Win max number \todo{10?}
		\begin{itemize}
			\item{Keep all drafted Monster}
			\item{Win re-rolling items (as outlined in the Items section)}
			\item{Get choice of 1 of 3 Colosseum-exclusive Monsters (which is the only way to obtain these Monsters)}
		\end{itemize}
	}
\end{itemize}

\postamble{}