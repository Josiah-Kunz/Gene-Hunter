% ==================
% Standalone support
% ==================


\makeatletter
\ifdefined\preloaded
\else
	\newcommand{\preloaded}{not set!}
	\ifx\documentclass\@twoclasseserror % after \documentclass
		\renewcommand{\preloaded}{true}
	\else
		\documentclass[12pt]{report}

% Packages
% ========
\usepackage[letterpaper, portrait, margin=1in]{geometry}
\usepackage{graphicx}			% So we can load figures with "." in their filename
\usepackage{float}				% So we can use "[H]"
\usepackage[dvipsnames]{xcolor}	% For custom colors
\usepackage{xparse}				% For \DeclareDocumentCommand
\usepackage{caption}			% For \captionof command
\usepackage{mathtools}			% For \DeclarePairedDelimeter
\usepackage{ifthen}				% For \ifthenelse{}{}{}
\usepackage{soul}				% For underlining with \ul
\setuldepth{x}					%	""
\usepackage{multicol}			% For multiple columns via \begin{multicols}{2}
\usepackage[most]{tcolorbox}	% For the tl; dr environment
\usepackage{array}				% For extended column definitions
\usepackage{tabularray}			% For \begin{longtblr} tables that span multiple columns and pages
\usepackage{amssymb}			% For $\checkmark$
\usepackage{etoc}				% For \localtableofcontents


% Bibliography
% ============
\usepackage[american]{babel}
\usepackage{csquotes}
\usepackage[style=ieee, backend=biber]{biblatex}
\usepackage{hyperref}
\hypersetup{colorlinks=true, allcolors=black, urlcolor=blue}
\addbibresource{../sources.bib}

% etoc setup
% ==========

\etocsetstyle{section}{}{}{\etocsavedchaptertocline{\numberline{}\etocname}{\etocpage}}{}
\etocsetstyle{subsection}{}{}{\etocsavedsectiontocline{\numberline{}\etocname}{\etocpage}}{}
\etocsetstyle{subsubsection}{}{}{\etocsavedsubsectiontocline{\numberline{}\etocname}{\etocpage}}{}
\etocsetstyle{paragraph}{}{}{\etocsavedsubsubsectiontocline{\numberline{}\etocname}{\etocpage}}{}
\etocsetstyle{subparagraph}{}{}{\etocsavedparagraphtocline{\numberline{}\etocname}{\etocpage}}{}


% Simple custom commands
% ======================
\makeatletter
\def\maxwidth#1{\ifdim\Gin@nat@width>#1 #1\else\Gin@nat@width\fi} 
\makeatother

\def\todo#1{\selectfont{\color{red}\texttt{\textbf{TODO:} #1}}}

% Sections 'n' such
% =================

\definecolor{chaptColor}{RGB}{0, 83, 161}

\def\chapt#1{
%
	% Default chapter behavior
	\begingroup\color{chaptColor}
	\chapter{#1}
	\endgroup
	
	% Label
	\label{chp:#1}
}

\definecolor{sectColor}{rgb}{0, 0.5, 0.0}
\def\sect#1{\textcolor{sectColor}{\section{#1}}}

\definecolor{subsectColor}{rgb}{0, 0.5, 0.5}
\def\subsect#1{\textcolor{subsectColor}{\subsection{#1}}\noindent}

\definecolor{subsubsectColor}{rgb}{0.747, 0.458, 0}
\def\subsubsect#1{\textcolor{subsubsectColor}{\subsubsection{#1}}}

% TL; DR section
% ==============
\newenvironment{tldr}{\begin{tcolorbox}[colback=gray!20!white,colframe=blue!75!black,title=TL; DR]}{\end{tcolorbox}\vspace*{12pt}}

% Custom math commands
% ====================
\DeclarePairedDelimiter\ceil{\lceil}{\rceil}
\DeclarePairedDelimiter\floor{\lfloor}{\rfloor}

% ======================
% \graphic{filename=...}
% ======================
% Keyword arguments
\makeatletter
\define@key{graphicKeys}{filename}{\def\graphic@filename{#1}}
\define@key{graphicKeys}{scale}{\def\graphic@scale{#1}}
\define@key{graphicKeys}{width}{\def\graphic@width{#1}}
\define@key{graphicKeys}{caption}{\def\graphic@caption{#1}}
\define@key{graphicKeys}{captionType}{\def\graphic@captionType{#1}}
\define@key{graphicKeys}{label}{\def\graphic@label{#1}}

% Kwargs
\DeclareDocumentCommand{\graphic}{m}{

	\begingroup
	
	% Set default kwargs
	\setkeys{graphicKeys}{filename={0}, #1}
	\setkeys{graphicKeys}{scale={0}, #1}
	\setkeys{graphicKeys}{width={0}, #1}
	\setkeys{graphicKeys}{caption={}, #1}
	\setkeys{graphicKeys}{captionType={figure}, #1}
	\setkeys{graphicKeys}{label={}, #1}
	
	% Assign width
	\let\graphicWidth\relax % let \mytmplen to \relax
	\newlength{\graphicWidth}
	\setlength{\graphicWidth}{\columnwidth}
	
	\if \graphic@scale 0
		
		\if \graphic@width 0
			
			\setlength{\graphicWidth}{0.5\columnwidth}
		
		\else
	
			\setlength{\graphicWidth}{\graphic@width}
			
		\fi

	\else
		
		\setlength{\graphicWidth}{\columnwidth * \graphic@scale}

	\fi	
	
	% Do figure
	\begin{minipage}{\columnwidth}
		\vspace*{12pt}
		\begin{center}
		
			\includegraphics[width = \graphicWidth]{\graphic@filename}
			
			\ifthenelse{\equal{\graphic@caption}{}}{}{
				\captionof{\graphic@captionType}{\graphic@caption}
			}
			
			\ifthenelse{\equal{\graphic@label}{}}{
				\label{fig: \graphic@filename}
			}{
				\label{\graphic@label}
			}
		\end{center}
		\vspace*{12pt}
	\end{minipage}
	
	\endgroup
}
\makeatother

% ==============
% Base Pair Page
% ==============
% Keyword arguments
\makeatletter
\define@key{bpPageKeys}{baseFilename}{\def\bpPage@baseFilename{#1}}
\define@key{bpPageKeys}{scale}{\def\bpPage@scale{#1}}
\define@key{bpPageKeys}{width}{\def\bpPage@width{#1}}
\define@key{bpPageKeys}{caption}{\def\bpPage@caption{#1}}
\define@key{bpPageKeys}{captionType}{\def\bpPage@captionType{#1}}
\define@key{bpPageKeys}{label}{\def\bpPage@label{#1}}

% Kwargs
\DeclareDocumentCommand{\BPPage}{m}{

	\begingroup
	
	% Set default kwargs
	\setkeys{bpPageKeys}{baseFilename={0}, #1}
	\setkeys{bpPageKeys}{caption={}, #1}
	\setkeys{bpPageKeys}{label={}, #1}
	
	\newpage

	\begin{center}
		\ul{\mbox{\bpPage@baseFilename{}: 1 Base Pair [min]}}
	\end{center}
	\vspace*{-36pt}
	\begin{minipage}[b]{0.5\textwidth}
		\graphic{filename=\bpPage@baseFilename-overview-1-table, width=\textwidth}
	\end{minipage}
	\begin{minipage}[b]{0.5\textwidth}
		\graphic{filename=\bpPage@baseFilename-overview-1-plot, width=\textwidth}
	\end{minipage}

	\begin{center}
		\ul{\mbox{\bpPage@baseFilename{}: 50 Base Pairs [average]}}
	\end{center}
	\vspace*{-36pt}
	\begin{minipage}[b]{0.5\textwidth}
		\graphic{filename=\bpPage@baseFilename-overview-50-table, width=\textwidth}
	\end{minipage}
	\begin{minipage}[b]{0.5\textwidth}
		\graphic{filename=\bpPage@baseFilename-overview-50-plot, width=\textwidth}
	\end{minipage}
	
	\begin{center}
		\ul{\mbox{\bpPage@baseFilename{}: 100 Base Pairs [max]}}
	\end{center}
	\vspace*{-36pt}
	\begin{minipage}[b]{0.5\textwidth}
		\graphic{filename=\bpPage@baseFilename-overview-100-table, width=\textwidth}
	\end{minipage}
	\begin{minipage}[b]{0.5\textwidth}
		\graphic{filename=\bpPage@baseFilename-overview-100-plot, width=\textwidth}
	\end{minipage}

	\vspace*{-24pt}
	\captionof{figure}{\bpPage@caption}\label{\bpPage@label}	
	
	\endgroup
}
\makeatother



\begin{document}
  		\begin{document}
		\renewcommand{\preloaded}{false}
	\fi
\fi
\makeatother
\trysetmain{}

% Additional usage: \ifthenelse{\equal{\preloaded}{true}}{ ... }{ ... }
% Note: Cannot do \trysetmain{} here because the \currfilename will always be "standalone.tex"

% ===========================================================
% Outlet cell function. Usage inside OutletTable environment:
%	\OutletCell{title= ..., 
%				beforeParams={ ... }, 
%				beforeNote={ Optional! },
%				afterParams={\code{const float OriginalYield},\\
%							 \code{const float ReturnedYield}
%							},
%				afterNote={ Also optional! }
%				}\\
% ===========================================================

% Keyword arguments
\makeatletter
\define@key{outletKeys}{title}{\def\outletKeys@title{#1}}
\define@key{outletKeys}{beforeParams}{\def\outletKeys@beforeParams{#1}}
\define@key{outletKeys}{afterParams}{\def\outletKeys@afterParams{#1}}
\define@key{outletKeys}{beforeNote}{\def\outletKeys@beforeNote{#1}}
\define@key{outletKeys}{afterNote}{\def\outletKeys@afterNote{#1}}

% Kwargs
\DeclareDocumentCommand{\OutletCell}{m}{

	\begingroup
	
	% Set default kwargs
	\setkeys{outletKeys}{title={0}, #1}
	\setkeys{outletKeys}{beforeParams={0}, #1}
	\setkeys{outletKeys}{afterParams={0}, #1}
	\setkeys{outletKeys}{beforeNote={}, #1}
	\setkeys{outletKeys}{afterNote={}, #1}

	% Title
	\begin{tblr}{
		colspec={Q[l, wd=\linewidth]},
		leftsep = 0pt,
		rightsep = 0pt,
		hline{1} = {2pt},
		hline{2} = {1.5pt},
	}
	\color{ICBlue}\textbf{\hspace*{0.5em}\outletKeys@title}
	\end{tblr}	
	
	% Before
	\begin{tblr}{
		colspec={Q[l, wd=0.2\linewidth] Q[l, wd=0.8\linewidth]},
		leftsep = 0pt,
		rightsep = 0pt,
	}
		\hspace*{1em}$\blacktriangleright$ Before	& {	\outletKeys@beforeParams }
	\end{tblr}
	
	% Before note
	\ifthenelse{\equal{\outletKeys@beforeNote}{}}{}{%
		\begin{tblr}{
			colspec={Q[l, wd=0.2\linewidth] Q[l, wd=0.8\linewidth]},
			leftsep = 0pt,
			rightsep = 0pt,
		}
			\hspace*{2em} \textit{Note:} & \outletKeys@beforeNote
		\end{tblr}
	}
	
	% After
	\begin{tblr}{
		colspec={Q[l, wd=0.2\linewidth] Q[l, wd=0.8\linewidth]},
		leftsep = 0pt,
		rightsep = 0pt,
		hline{1} = 1pt,
	}
		\hspace*{1em}$\blacktriangleright$ After	& {	\outletKeys@afterParams }
	\end{tblr}
	
	% After note
	\ifthenelse{\equal{\outletKeys@afterNote}{}}{}{%
		\begin{tblr}{
			colspec={Q[l, wd=0.2\linewidth] Q[l, wd=0.8\linewidth]},
			leftsep = 0pt,
			rightsep = 0pt,
		}
			\hspace*{2em} \textit{Note:} & \outletKeys@afterNote
		\end{tblr}
	}
	
	\endgroup
}
\makeatother

% ================================
% Outlet table environment. Usage:
%	\begin{OutletTable}{Title}
%		...
%		[Use \OutletCell}
%		...
%	\end{OutletTable}
% ================================
\newenvironment{OutletTable}[1]{%
\begin{longtblr}[
	caption = {\code{Outlet}s for \code{#1}},
	label = {delegate-arrays-levelcomponent},
]{
	cells = {gray!15},
	leftsep = 0pt,
	rightsep = 0pt,
	colspec={|Q[c, wd=\linewidth]|},
	hline{2-Z}={2pt},
	%rowsep=-0.5em,
	rows = {abovesep=-12pt},
}
}{%
\end{longtblr}
}



%--------------------------

\begin{OutletTable}{AffinitiesComponent}

	\OutletCell{title=GetUnspentPoints, 
				beforeParams={\code{const uint8 OriginalPoints},\\
							\code{uint8\& ReturnedPoints}},
				afterParams={\code{const uint8 OriginalPoints},\\
							\code{const uint8 ReturnedPoints}}
				}
				\\
				
	\OutletCell{title=SetUnspentPoints, 
				beforeParams={\code{const uint8 OriginalPoints},\\
							\code{const uint8 InputPoints},\\
							\code{uint8\& SetPoints}},
				afterParams={\code{const uint8 OriginalPoints},\\
							\code{const uint8 InputPoints},\\
							\code{const uint8 SetPoints}}
				}
				\\
	
\end{OutletTable}

%--------------------------

\gap{}

\begin{OutletTable}{EffectComponent}

	\OutletCell{title=GetStacks, 
				beforeParams={\code{const uint16 OGStacks},\\
							\code{int32\& ReturnedStacks}},
				afterParams={\code{const uint16 OGStacks},\\
							\code{const int32 ReturnedStacks}},
				}
				\\
				
	\OutletCell{title=OnAddEffect, 
				beforeParams={\code{const EffectComponent* EffectToAdd}
							},
				afterParams={\code{const EffectComponent* AddedEffect}
							},
				}
				\\
				
	\OutletCell{title=OnRemoveEffect, 
				beforeParams={\code{const EffectComponent* EffectToRemove}
							},
				afterParams={\code{const EffectComponent* RemovedEffect}
							},
				}
				\\
	
\end{OutletTable}

%--------------------------

\gap{}

\begin{OutletTable}{LevelComponent}

	\OutletCell{title=GetBaseExpYield, 
				beforeParams={\code{const float OriginalYield},\\
							\code{float\& ReturnedYield}},
				afterParams={\code{const float OriginalYield},\\
							\code{const float ReturnedYield}}
				}
				\\
				
	% ----------------------------------------
				
	\OutletCell{title=GetCXP, 
				beforeParams={\code{const uint32 OriginalCXP},\\ 
							\code{int32\& ReturnedCXP}},
				beforeNote={\code{ReturnedCXP} is \code{int32\&} instead of \code{uint32\&} for Blueprint compatability.},
				afterParams={\code{const uint32 OriginalCXP}\\ 
							\code{const int32 ReturnedCXP}},
				afterNote={\code{ReturnedCXP} is \code{const int32} instead of \code{const uint32} for Blueprint compatability.}
				}	
				\\			

	% ----------------------------------------				
				
	\OutletCell{title=GetExpYield, 
				beforeParams={\code{const float OriginalYield},\\ 
							\code{float\& ReturnedYield},\\
							\code{const uint16 DefeatedLevel},\\
							\code{const uint16 VictoriousLevel}},
				beforeNote={``Defeated'' and ``Victorious'' levels are provided for flexibility (e.g., in case you want to yield exp differently based on level difference, although technically you could always back-calculate the level difference based on the equation and \code{OriginalYield}).},
				afterParams={\code{const float OriginalYied},\\ 
							\code{const float ReturnedYield},\\
							\code{const uint16 DefeatedLevel},\\
							\code{const uint16 VictoriousLevel}},
				afterNote={``Defeated'' and ``Victorious'' levels are provided for symmetry with respect to the \code{Before} delegate (since \code{ReturnedValue} is already calculated, I can't think of why you would need them, but you never know!).}
				}
				\\
			
	% ----------------------------------------			
				
	\OutletCell{title=GetMaxLevel, 
				beforeParams={\code{const uint16 DefaultMax},\\ 
							\code{int32\& AttemptedMax}},
				beforeNote={\code{DefaultMax} is defined in the code. It should normally be 100, but may change for certain subclasses (e.g., a \code{BossLevelComponent} may have a max of 200 instead). \\ 
				\code{AttemptedMax} is \code{int32\&} instead of \code{uint16\&} for Blueprint compatability.
				},
				afterParams={\code{const uint16 DefaultMax}\\ 
							\code{const int32 ReturnedMax}},
				}
				\\

	% ----------------------------------------

	\OutletCell{title=GetMinLevel, 
				beforeParams={\code{const uint16 DefaultMin},\\ 
							\code{int32\& AttemptedMin}},
				beforeNote={\code{DefaultMin} is defined in the code. It should normally be 1, but may change for certain subclasses (e.g., a \code{EggLevelComponent} may have a min of 0 instead for whatever reason).\\
				\code{AttemptedMin} is \code{int32\&} instead of \code{uint16\&} for Blueprint compatability.},
				afterParams={\code{const uint16 DefaultMin}\\ 
							\code{const int32 ReturnedMin}},
				afterNote={\code{ReturnedCXP} is \code{const int32} instead of \code{const uint32} for Blueprint compatability.}
				}
				\\
		
	% ----------------------------------------		
				
	\OutletCell{title=GetBaseExpYield, 
				beforeParams={\code{const float OriginalYield},\\
							\code{float\& ReturnedYield}},
				afterParams={\code{const float OriginalYield},\\
							\code{const float ReturnedYield}}
				}
				\\
		
	% ----------------------------------------		
				
	\OutletCell{title=SetBaseExpYield, 
				beforeParams={\code{const float OldYield},\\
							\code{const float InputYield},\\ 
							\code{float\& AttemptedYield}},
				afterParams={\code{const float OldYield}\\
							\code{const float InputYield},\\ 
							\code{const float NewYield}},
				afterNote={$\triangleright$ \code{OldYield} is the yield prior to calling \code{SetBaseExpYield},\\
							$\triangleright$ \code{InputYield} is the original, unmodified input to \code{SetBaseExpYield},\\
							$\triangleright$ \code{AttemptedYield} is the modified value that will be used to set the base exp yield.
							}
				}
				\\
		
	% ----------------------------------------		
						
	\OutletCell{title=SetCXP, 
				beforeParams={\code{const uint32 OldCXP},\\ 
							\code{const int32 InputCXP},\\
							\code{int32\& AttemptedCXP}},
				beforeNote={\code{AttemptedCXP} is \code{int32\&} instead of \code{uint32\&} for Blueprint compatability.},
				afterParams={\code{const uint32 OldCXP}\\ 
							\code{const int32 InputCXP},\\
							\code{const uint32 NewCXP}},
				afterNote={\code{StatsComponent} subscribes to this in order to change stats on level change.\\
							$\triangleright$ \code{OldCXP} is the cumulative experience points prior to calling \code{SetCXP},\\
							$\triangleright$ \code{InputCXP} is the original, unmodified input to \code{SetCXP},\\
							$\triangleright$ \code{AttemptedCXP} is the modified value that will be used to set the cumulative experience points.		
					}
				}
				\\
		
\end{OutletTable}

%--------------------------

\gap{}

\begin{OutletTable}{StatsComponent}

	% ----------------------------------------		
						
	\OutletCell{title=ApplyDamage, 
				beforeParams={
							\code{float\& BasePower},\\
							\code{float\& CritMultiplier},\\
							\code{float\& RandFluct},\\
							\code{float\& Stab},\\
							\code{float\& TypeAdvantage},\\
							\code{UCombatStatsComponent* Attacker},\\
							\code{UCombatStatsComponent* OwningStats}
							},
				beforeNote={Other quantities, such as the Attacker's attacking Stat, may be calculated from the given quantities.},
				afterParams={
							\code{const float BasePower},\\
							\code{const float CritMultiplier},\\
							\code{const float RandFluct},\\
							\code{const float\& Stab},\\
							\code{const float\& TypeAdvantage},\\
							\code{UCombatStatsComponent* Attacker},\\
							\code{UCombatStatsComponent* OwningStats}
							}
				}
				\\

	% ----------------------------------------		
						
	\OutletCell{title=CalculateDamage, 
				beforeParams={
							\code{float\& BasePower},\\
							\code{float\& CritMultiplier},\\
							\code{float\& RandFluct},\\
							\code{float\& Stab},\\
							\code{float\& TypeAdvantage},\\
							\code{UCombatStatsComponent* Attacker},\\
							\code{UCombatStatsComponent* OwningStats}
							},
				afterParams={
							\code{const float BasePower},\\
							\code{const float CritMultiplier},\\
							\code{const float RandFluct},\\
							\code{const float\& Stab},\\
							\code{const float\& TypeAdvantage},\\
							\code{UCombatStatsComponent* Attacker},\\
							\code{UCombatStatsComponent* OwningStats}
							},
				afterNote={The difference between calculating damage and applying damage is theoretical. For example, low-level AI might use \code{CalculateDamage} to make decisions. On the other hand, applying the damage might invoke some kind of reaction, like raising Physical Attack if hit by a move it's weak to.}
				}
				\\

	% ----------------------------------------		
						
	\OutletCell{title=GetCritMult, 
				beforeParams={\code{float\& BaseMultiplier},\\
							\code{float\& CritBonus},\\
							\code{UCombatStatsComponent* OwningStats}
							},
				beforeNote={Total crit bonus is \code{BaseMultiplier} + \code{CritBonus}; for example, 1.5 + 0.2.},
				afterParams={\code{const float BaseMultiplier},\\
							\code{const float CritBonus},\\
							\code{UCombatStatsComponent* OwningStats}
							}
				}
				\\

	% ----------------------------------------		
						
	\OutletCell{title=ModifyStat, 
				beforeParams={\code{const EStatEnum TargetStat},\\
							\code{const EStatValueType ValueType},\\
							\code{const EModificationMode Mode},\\
							\code{const float OriginalValue},\\
							\code{float\& AttemptedValue}
							},
				afterParams={\code{const EStatEnum TargetStat},\\
							\code{const EStatValueType ValueType},\\
							\code{const EModificationMode Mode},\\
							\code{const float OriginalValue},\\
							\code{const float NewValue}
							},
				afterNote={All ``ModifyStat'' functions from \code{StatsComponent} (such as \code{ModifyStatsUniformly} or \code{RandomizeStats}) go through \code{ModifyStatInternal}, which calls this \code{Outlet}.}
				}
				\\

	% ----------------------------------------		
						
	\OutletCell{title=RandomizeStats, 
				beforeParams={\code{const EStatEnum TargetStat},\\
							\code{const FStatRandParams OriginalParams},\\
							\code{FStatRandParams\&  ParamsToBeUsed}},
				afterParams={\code{const EStatEnum TargetStat},\\
							\code{const FStatRandParams OriginalParams},\\
							\code{const FStatRandParams UsedParams}},
				afterNote={The \code{EStatEnum} is not the acutal \code{FStat}. To get the \code{FStat} (such as \code{FHealth}), use \code{StatsComponent::GetStat(EStatEnum)}}
				}
				\\
		
	% ----------------------------------------		
						
	\OutletCell{title=RecalculateStats, 
				beforeParams={\code{const EStatEnum TargetStat},\\
							\code{const bool bResetCurrent},\\
							\code{const float OriginalCurrent},\\
							\code{const float OriginalPermanent}},
				afterParams={\code{const EStatEnum TargetStat},\\
							\code{const bool bResetCurrent},\\
							\code{const float OriginalCurrent},\\
							\code{const float OriginalPermanent}},
				}
				\\
			
		
	
\end{OutletTable}


\postamble{}