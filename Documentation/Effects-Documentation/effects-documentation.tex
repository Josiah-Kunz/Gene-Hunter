% ==================
% Standalone support
% ==================


\makeatletter
\ifdefined\preloaded
\else
	\newcommand{\preloaded}{not set!}
	\ifx\documentclass\@twoclasseserror % after \documentclass
		\renewcommand{\preloaded}{true}
	\else
		\documentclass[12pt]{report}

% Packages
% ========
\usepackage[letterpaper, portrait, margin=1in]{geometry}
\usepackage{graphicx}			% So we can load figures with "." in their filename
\usepackage{float}				% So we can use "[H]"
\usepackage[dvipsnames]{xcolor}	% For custom colors
\usepackage{xparse}				% For \DeclareDocumentCommand
\usepackage{caption}			% For \captionof command
\usepackage{mathtools}			% For \DeclarePairedDelimeter
\usepackage{ifthen}				% For \ifthenelse{}{}{}
\usepackage{soul}				% For underlining with \ul
\setuldepth{x}					%	""
\usepackage{multicol}			% For multiple columns via \begin{multicols}{2}
\usepackage[most]{tcolorbox}	% For the tl; dr environment
\usepackage{array}				% For extended column definitions
\usepackage{tabularray}			% For \begin{longtblr} tables that span multiple columns and pages
\usepackage{amssymb}			% For $\checkmark$
\usepackage{etoc}				% For \localtableofcontents


% Bibliography
% ============
\usepackage[american]{babel}
\usepackage{csquotes}
\usepackage[style=ieee, backend=biber]{biblatex}
\usepackage{hyperref}
\hypersetup{colorlinks=true, allcolors=black, urlcolor=blue}
\addbibresource{../sources.bib}

% etoc setup
% ==========

\etocsetstyle{section}{}{}{\etocsavedchaptertocline{\numberline{}\etocname}{\etocpage}}{}
\etocsetstyle{subsection}{}{}{\etocsavedsectiontocline{\numberline{}\etocname}{\etocpage}}{}
\etocsetstyle{subsubsection}{}{}{\etocsavedsubsectiontocline{\numberline{}\etocname}{\etocpage}}{}
\etocsetstyle{paragraph}{}{}{\etocsavedsubsubsectiontocline{\numberline{}\etocname}{\etocpage}}{}
\etocsetstyle{subparagraph}{}{}{\etocsavedparagraphtocline{\numberline{}\etocname}{\etocpage}}{}


% Simple custom commands
% ======================
\makeatletter
\def\maxwidth#1{\ifdim\Gin@nat@width>#1 #1\else\Gin@nat@width\fi} 
\makeatother

\def\todo#1{\selectfont{\color{red}\texttt{\textbf{TODO:} #1}}}

% Sections 'n' such
% =================

\definecolor{chaptColor}{RGB}{0, 83, 161}

\def\chapt#1{
%
	% Default chapter behavior
	\begingroup\color{chaptColor}
	\chapter{#1}
	\endgroup
	
	% Label
	\label{chp:#1}
}

\definecolor{sectColor}{rgb}{0, 0.5, 0.0}
\def\sect#1{\textcolor{sectColor}{\section{#1}}}

\definecolor{subsectColor}{rgb}{0, 0.5, 0.5}
\def\subsect#1{\textcolor{subsectColor}{\subsection{#1}}\noindent}

\definecolor{subsubsectColor}{rgb}{0.747, 0.458, 0}
\def\subsubsect#1{\textcolor{subsubsectColor}{\subsubsection{#1}}}

% TL; DR section
% ==============
\newenvironment{tldr}{\begin{tcolorbox}[colback=gray!20!white,colframe=blue!75!black,title=TL; DR]}{\end{tcolorbox}\vspace*{12pt}}

% Custom math commands
% ====================
\DeclarePairedDelimiter\ceil{\lceil}{\rceil}
\DeclarePairedDelimiter\floor{\lfloor}{\rfloor}

% ======================
% \graphic{filename=...}
% ======================
% Keyword arguments
\makeatletter
\define@key{graphicKeys}{filename}{\def\graphic@filename{#1}}
\define@key{graphicKeys}{scale}{\def\graphic@scale{#1}}
\define@key{graphicKeys}{width}{\def\graphic@width{#1}}
\define@key{graphicKeys}{caption}{\def\graphic@caption{#1}}
\define@key{graphicKeys}{captionType}{\def\graphic@captionType{#1}}
\define@key{graphicKeys}{label}{\def\graphic@label{#1}}

% Kwargs
\DeclareDocumentCommand{\graphic}{m}{

	\begingroup
	
	% Set default kwargs
	\setkeys{graphicKeys}{filename={0}, #1}
	\setkeys{graphicKeys}{scale={0}, #1}
	\setkeys{graphicKeys}{width={0}, #1}
	\setkeys{graphicKeys}{caption={}, #1}
	\setkeys{graphicKeys}{captionType={figure}, #1}
	\setkeys{graphicKeys}{label={}, #1}
	
	% Assign width
	\let\graphicWidth\relax % let \mytmplen to \relax
	\newlength{\graphicWidth}
	\setlength{\graphicWidth}{\columnwidth}
	
	\if \graphic@scale 0
		
		\if \graphic@width 0
			
			\setlength{\graphicWidth}{0.5\columnwidth}
		
		\else
	
			\setlength{\graphicWidth}{\graphic@width}
			
		\fi

	\else
		
		\setlength{\graphicWidth}{\columnwidth * \graphic@scale}

	\fi	
	
	% Do figure
	\begin{minipage}{\columnwidth}
		\vspace*{12pt}
		\begin{center}
		
			\includegraphics[width = \graphicWidth]{\graphic@filename}
			
			\ifthenelse{\equal{\graphic@caption}{}}{}{
				\captionof{\graphic@captionType}{\graphic@caption}
			}
			
			\ifthenelse{\equal{\graphic@label}{}}{
				\label{fig: \graphic@filename}
			}{
				\label{\graphic@label}
			}
		\end{center}
		\vspace*{12pt}
	\end{minipage}
	
	\endgroup
}
\makeatother

% ==============
% Base Pair Page
% ==============
% Keyword arguments
\makeatletter
\define@key{bpPageKeys}{baseFilename}{\def\bpPage@baseFilename{#1}}
\define@key{bpPageKeys}{scale}{\def\bpPage@scale{#1}}
\define@key{bpPageKeys}{width}{\def\bpPage@width{#1}}
\define@key{bpPageKeys}{caption}{\def\bpPage@caption{#1}}
\define@key{bpPageKeys}{captionType}{\def\bpPage@captionType{#1}}
\define@key{bpPageKeys}{label}{\def\bpPage@label{#1}}

% Kwargs
\DeclareDocumentCommand{\BPPage}{m}{

	\begingroup
	
	% Set default kwargs
	\setkeys{bpPageKeys}{baseFilename={0}, #1}
	\setkeys{bpPageKeys}{caption={}, #1}
	\setkeys{bpPageKeys}{label={}, #1}
	
	\newpage

	\begin{center}
		\ul{\mbox{\bpPage@baseFilename{}: 1 Base Pair [min]}}
	\end{center}
	\vspace*{-36pt}
	\begin{minipage}[b]{0.5\textwidth}
		\graphic{filename=\bpPage@baseFilename-overview-1-table, width=\textwidth}
	\end{minipage}
	\begin{minipage}[b]{0.5\textwidth}
		\graphic{filename=\bpPage@baseFilename-overview-1-plot, width=\textwidth}
	\end{minipage}

	\begin{center}
		\ul{\mbox{\bpPage@baseFilename{}: 50 Base Pairs [average]}}
	\end{center}
	\vspace*{-36pt}
	\begin{minipage}[b]{0.5\textwidth}
		\graphic{filename=\bpPage@baseFilename-overview-50-table, width=\textwidth}
	\end{minipage}
	\begin{minipage}[b]{0.5\textwidth}
		\graphic{filename=\bpPage@baseFilename-overview-50-plot, width=\textwidth}
	\end{minipage}
	
	\begin{center}
		\ul{\mbox{\bpPage@baseFilename{}: 100 Base Pairs [max]}}
	\end{center}
	\vspace*{-36pt}
	\begin{minipage}[b]{0.5\textwidth}
		\graphic{filename=\bpPage@baseFilename-overview-100-table, width=\textwidth}
	\end{minipage}
	\begin{minipage}[b]{0.5\textwidth}
		\graphic{filename=\bpPage@baseFilename-overview-100-plot, width=\textwidth}
	\end{minipage}

	\vspace*{-24pt}
	\captionof{figure}{\bpPage@caption}\label{\bpPage@label}	
	
	\endgroup
}
\makeatother



\begin{document}
  		\begin{document}
		\renewcommand{\preloaded}{false}
	\fi
\fi
\makeatother
\trysetmain{}

% Additional usage: \ifthenelse{\equal{\preloaded}{true}}{ ... }{ ... }
% Note: Cannot do \trysetmain{} here because the \currfilename will always be "standalone.tex"

\begin{tldr}
	\todo{}
\end{tldr}

%====================================================

\sect{Structure}

\newcommand{\SubItem}[1]{\begin{itemize}\item{#1}\end{itemize}}
\newcommand{\ScreenshotScale}{2.5}

\begin{itemize}
	\item{\textbf{\code{EffectableComponent}s} are \code{ActorComponent}s that allow for delegation (effects). They have predefined places called ``\code{Outlet}s'' that allow for code modification. Think of \code{Outlet}s like electrical outlets waiting to be plugged into.
		\SubItem{Let's use \code{StatsComponent} as an example. Say we want a Pok\'{e}mon-style ``Adamant'' nature ($+$10\% PhA/$-$10\%SpA). One such place for modification is in the function \code{RecalculateStats}. \todo{Update picture!}}
		\begin{center}
			\includegraphics[scale=\ScreenshotScale]{recalculate-stats-code}
		\end{center}
		}
	\item{\textbf{\code{Outlet} arrays} are variables inside of \code{EffectableComponent}s. They hold \code{Outlet}s whose delegates execute when needed.
		\SubItem{\todo{Update this!} Let's use \code{StatsComponent}'s \code{AfterRecalculateStatsArray} in our example. In this case, after stats are recalculated (say, on level-up), the base PhA would increase by 10\% and the base SpA would decrease by 10\% (additively): }
		\begin{center}
			\includegraphics[scale=\ScreenshotScale]{adamant-code}
		\end{center}		 
		}
	\item{\textbf{\code{EffectComponent}s} are \code{ActorComponent}s that plug into \code{Outlet}s. These come in many forms, but an easy example is a \code{Buff}. \todo{Describe how this happens with pictures!}}
\end{itemize}

%====================================================

\sect{\code{EffectComponent} Inheritance}

% ==================
% Standalone support
% ==================


\makeatletter
\ifdefined\preloaded
\else
	\newcommand{\preloaded}{not set!}
	\ifx\documentclass\@twoclasseserror % after \documentclass
		\renewcommand{\preloaded}{true}
	\else
		\documentclass[12pt]{report}

% Packages
% ========
\usepackage[letterpaper, portrait, margin=1in]{geometry}
\usepackage{graphicx}			% So we can load figures with "." in their filename
\usepackage{float}				% So we can use "[H]"
\usepackage[dvipsnames]{xcolor}	% For custom colors
\usepackage{xparse}				% For \DeclareDocumentCommand
\usepackage{caption}			% For \captionof command
\usepackage{mathtools}			% For \DeclarePairedDelimeter
\usepackage{ifthen}				% For \ifthenelse{}{}{}
\usepackage{soul}				% For underlining with \ul
\setuldepth{x}					%	""
\usepackage{multicol}			% For multiple columns via \begin{multicols}{2}
\usepackage[most]{tcolorbox}	% For the tl; dr environment
\usepackage{array}				% For extended column definitions
\usepackage{tabularray}			% For \begin{longtblr} tables that span multiple columns and pages
\usepackage{amssymb}			% For $\checkmark$
\usepackage{etoc}				% For \localtableofcontents


% Bibliography
% ============
\usepackage[american]{babel}
\usepackage{csquotes}
\usepackage[style=ieee, backend=biber]{biblatex}
\usepackage{hyperref}
\hypersetup{colorlinks=true, allcolors=black, urlcolor=blue}
\addbibresource{../sources.bib}

% etoc setup
% ==========

\etocsetstyle{section}{}{}{\etocsavedchaptertocline{\numberline{}\etocname}{\etocpage}}{}
\etocsetstyle{subsection}{}{}{\etocsavedsectiontocline{\numberline{}\etocname}{\etocpage}}{}
\etocsetstyle{subsubsection}{}{}{\etocsavedsubsectiontocline{\numberline{}\etocname}{\etocpage}}{}
\etocsetstyle{paragraph}{}{}{\etocsavedsubsubsectiontocline{\numberline{}\etocname}{\etocpage}}{}
\etocsetstyle{subparagraph}{}{}{\etocsavedparagraphtocline{\numberline{}\etocname}{\etocpage}}{}


% Simple custom commands
% ======================
\makeatletter
\def\maxwidth#1{\ifdim\Gin@nat@width>#1 #1\else\Gin@nat@width\fi} 
\makeatother

\def\todo#1{\selectfont{\color{red}\texttt{\textbf{TODO:} #1}}}

% Sections 'n' such
% =================

\definecolor{chaptColor}{RGB}{0, 83, 161}

\def\chapt#1{
%
	% Default chapter behavior
	\begingroup\color{chaptColor}
	\chapter{#1}
	\endgroup
	
	% Label
	\label{chp:#1}
}

\definecolor{sectColor}{rgb}{0, 0.5, 0.0}
\def\sect#1{\textcolor{sectColor}{\section{#1}}}

\definecolor{subsectColor}{rgb}{0, 0.5, 0.5}
\def\subsect#1{\textcolor{subsectColor}{\subsection{#1}}\noindent}

\definecolor{subsubsectColor}{rgb}{0.747, 0.458, 0}
\def\subsubsect#1{\textcolor{subsubsectColor}{\subsubsection{#1}}}

% TL; DR section
% ==============
\newenvironment{tldr}{\begin{tcolorbox}[colback=gray!20!white,colframe=blue!75!black,title=TL; DR]}{\end{tcolorbox}\vspace*{12pt}}

% Custom math commands
% ====================
\DeclarePairedDelimiter\ceil{\lceil}{\rceil}
\DeclarePairedDelimiter\floor{\lfloor}{\rfloor}

% ======================
% \graphic{filename=...}
% ======================
% Keyword arguments
\makeatletter
\define@key{graphicKeys}{filename}{\def\graphic@filename{#1}}
\define@key{graphicKeys}{scale}{\def\graphic@scale{#1}}
\define@key{graphicKeys}{width}{\def\graphic@width{#1}}
\define@key{graphicKeys}{caption}{\def\graphic@caption{#1}}
\define@key{graphicKeys}{captionType}{\def\graphic@captionType{#1}}
\define@key{graphicKeys}{label}{\def\graphic@label{#1}}

% Kwargs
\DeclareDocumentCommand{\graphic}{m}{

	\begingroup
	
	% Set default kwargs
	\setkeys{graphicKeys}{filename={0}, #1}
	\setkeys{graphicKeys}{scale={0}, #1}
	\setkeys{graphicKeys}{width={0}, #1}
	\setkeys{graphicKeys}{caption={}, #1}
	\setkeys{graphicKeys}{captionType={figure}, #1}
	\setkeys{graphicKeys}{label={}, #1}
	
	% Assign width
	\let\graphicWidth\relax % let \mytmplen to \relax
	\newlength{\graphicWidth}
	\setlength{\graphicWidth}{\columnwidth}
	
	\if \graphic@scale 0
		
		\if \graphic@width 0
			
			\setlength{\graphicWidth}{0.5\columnwidth}
		
		\else
	
			\setlength{\graphicWidth}{\graphic@width}
			
		\fi

	\else
		
		\setlength{\graphicWidth}{\columnwidth * \graphic@scale}

	\fi	
	
	% Do figure
	\begin{minipage}{\columnwidth}
		\vspace*{12pt}
		\begin{center}
		
			\includegraphics[width = \graphicWidth]{\graphic@filename}
			
			\ifthenelse{\equal{\graphic@caption}{}}{}{
				\captionof{\graphic@captionType}{\graphic@caption}
			}
			
			\ifthenelse{\equal{\graphic@label}{}}{
				\label{fig: \graphic@filename}
			}{
				\label{\graphic@label}
			}
		\end{center}
		\vspace*{12pt}
	\end{minipage}
	
	\endgroup
}
\makeatother

% ==============
% Base Pair Page
% ==============
% Keyword arguments
\makeatletter
\define@key{bpPageKeys}{baseFilename}{\def\bpPage@baseFilename{#1}}
\define@key{bpPageKeys}{scale}{\def\bpPage@scale{#1}}
\define@key{bpPageKeys}{width}{\def\bpPage@width{#1}}
\define@key{bpPageKeys}{caption}{\def\bpPage@caption{#1}}
\define@key{bpPageKeys}{captionType}{\def\bpPage@captionType{#1}}
\define@key{bpPageKeys}{label}{\def\bpPage@label{#1}}

% Kwargs
\DeclareDocumentCommand{\BPPage}{m}{

	\begingroup
	
	% Set default kwargs
	\setkeys{bpPageKeys}{baseFilename={0}, #1}
	\setkeys{bpPageKeys}{caption={}, #1}
	\setkeys{bpPageKeys}{label={}, #1}
	
	\newpage

	\begin{center}
		\ul{\mbox{\bpPage@baseFilename{}: 1 Base Pair [min]}}
	\end{center}
	\vspace*{-36pt}
	\begin{minipage}[b]{0.5\textwidth}
		\graphic{filename=\bpPage@baseFilename-overview-1-table, width=\textwidth}
	\end{minipage}
	\begin{minipage}[b]{0.5\textwidth}
		\graphic{filename=\bpPage@baseFilename-overview-1-plot, width=\textwidth}
	\end{minipage}

	\begin{center}
		\ul{\mbox{\bpPage@baseFilename{}: 50 Base Pairs [average]}}
	\end{center}
	\vspace*{-36pt}
	\begin{minipage}[b]{0.5\textwidth}
		\graphic{filename=\bpPage@baseFilename-overview-50-table, width=\textwidth}
	\end{minipage}
	\begin{minipage}[b]{0.5\textwidth}
		\graphic{filename=\bpPage@baseFilename-overview-50-plot, width=\textwidth}
	\end{minipage}
	
	\begin{center}
		\ul{\mbox{\bpPage@baseFilename{}: 100 Base Pairs [max]}}
	\end{center}
	\vspace*{-36pt}
	\begin{minipage}[b]{0.5\textwidth}
		\graphic{filename=\bpPage@baseFilename-overview-100-table, width=\textwidth}
	\end{minipage}
	\begin{minipage}[b]{0.5\textwidth}
		\graphic{filename=\bpPage@baseFilename-overview-100-plot, width=\textwidth}
	\end{minipage}

	\vspace*{-24pt}
	\captionof{figure}{\bpPage@caption}\label{\bpPage@label}	
	
	\endgroup
}
\makeatother



\begin{document}
  		\begin{document}
		\renewcommand{\preloaded}{false}
	\fi
\fi
\makeatother
\trysetmain{}

% Additional usage: \ifthenelse{\equal{\preloaded}{true}}{ ... }{ ... }
% Note: Cannot do \trysetmain{} here because the \currfilename will always be "standalone.tex"

\newcommand{\TreeNote}[1]{{\fontsize{7pt}{0pt} \selectfont{\color{gray} #1}}}

% auxiliary nodes without node label
\forestset{
  empty nodes/.style={
    delay={where content={}{shape=coordinate,for parent={for children={anchor=north}}}{}}}
}

The base classes inherit as:

\vspace*{1em}

\begin{center}
\fbox{\small \begin{forest}
before typesetting nodes={for tree={
	s sep=0cm,
	parent anchor=south, edge path={\noexpand\path [\forestoption{edge}] (!u.parent anchor) -- ++(0,-5pt) -| (.child anchor)\forestoption{edge label};},
	empty nodes
 	}}
[Effect\\\TreeNote{base class, stackable}
		[Aura\\\TreeNote{visible, purgeable,}\\\TreeNote{not timed}
			[Negative\\\TreeNote{``bad'' Aura}
			]
			[Positive\\\TreeNote{``good'' Aura}
			]
			[Zone\\\TreeNote{area of effect}\\\TreeNote{``good'' or ``bad''}\\\TreeNote{non-purgeable}
			]
		]
		[Intrinsic\\\TreeNote{non-visible,}\\\TreeNote{non-purgeable}
			[Mutation\\\TreeNote{random ability}\\\TreeNote{based on Species}
			]
			[{Type Trait}\\\TreeNote{based on Type}
			]
		] 
		[Timed\\\TreeNote{visible, purgeable,}\\\TreeNote{set duration}
			[Buff\\\TreeNote{``good'' Timed Effect}
				[HoT\\\TreeNote{\ul{h}eal \ul{o}ver \ul{t}ime}
				]
			]
			[Debuff\\\TreeNote{``bad'' Timed Effect}
				[DoT\\\TreeNote{\ul{d}amage \ul{o}ver \ul{t}ime}
				]
			]
		]
	]
\end{forest}}
\end{center}

\vspace*{1em}

Some notes:

\begin{itemize}
	\item{Only the base names have been used. That is, the actual names may be\\ \code{UTimedEffectComponent} instead of simply ``Timed''.}
	\item{``Purgeable'' means it is possible to reduce the stacks of the \code{UEffectComponent} down to zero (detachment of \code{UEffectComponent}).}
	\item{All \code{UEffectComponent}s are ``silenceable'', meaning their effects can be nullified (but not detached or reduced in stacks).}
\end{itemize}


\postamble{}



%====================================================

\sect{List of \code{EffectableComponent}s and \code{Outlet}s}

The following tables show all implemented \code{EffectableComponent}s and their delegate arrays. Note the ``base name'' indicates existence of ``Before'' and ``After'' versions of:

\begin{enumerate}
	\item{the delegate signatures, \code{FBeforeBaseNameSignature};}
	\item{the delegate wrappers, \code{FBeforeBaseNameDelegate}, which are necessary since \code{TArray}s cannot contain delegates;}
	\item{the private arrays of delegate wrappers,\\ \code{TArray<FBeforeBaseNameOutlet> BeforeDelegates};}
	\item{a function to execute the arrays, \code{ExecuteBeforeBaseName}; and}
	\item{\code{AddBeforeBaseName}, a function to add an \code{Outlet} to the private array \code{BeforeDelegates} (which also puts it in the right order based on priority).}
\end{enumerate}

\noindent Note that the philosophy applies to what is \textit{probable} rather than what is \textit{possible}. Hence the list meant to be practical rather than exhaustive.

%====================================================

% ==================
% Standalone support
% ==================


\makeatletter
\ifdefined\preloaded
\else
	\newcommand{\preloaded}{not set!}
	\ifx\documentclass\@twoclasseserror % after \documentclass
		\renewcommand{\preloaded}{true}
	\else
		\documentclass[12pt]{report}

% Packages
% ========
\usepackage[letterpaper, portrait, margin=1in]{geometry}
\usepackage{graphicx}			% So we can load figures with "." in their filename
\usepackage{float}				% So we can use "[H]"
\usepackage[dvipsnames]{xcolor}	% For custom colors
\usepackage{xparse}				% For \DeclareDocumentCommand
\usepackage{caption}			% For \captionof command
\usepackage{mathtools}			% For \DeclarePairedDelimeter
\usepackage{ifthen}				% For \ifthenelse{}{}{}
\usepackage{soul}				% For underlining with \ul
\setuldepth{x}					%	""
\usepackage{multicol}			% For multiple columns via \begin{multicols}{2}
\usepackage[most]{tcolorbox}	% For the tl; dr environment
\usepackage{array}				% For extended column definitions
\usepackage{tabularray}			% For \begin{longtblr} tables that span multiple columns and pages
\usepackage{amssymb}			% For $\checkmark$
\usepackage{etoc}				% For \localtableofcontents


% Bibliography
% ============
\usepackage[american]{babel}
\usepackage{csquotes}
\usepackage[style=ieee, backend=biber]{biblatex}
\usepackage{hyperref}
\hypersetup{colorlinks=true, allcolors=black, urlcolor=blue}
\addbibresource{../sources.bib}

% etoc setup
% ==========

\etocsetstyle{section}{}{}{\etocsavedchaptertocline{\numberline{}\etocname}{\etocpage}}{}
\etocsetstyle{subsection}{}{}{\etocsavedsectiontocline{\numberline{}\etocname}{\etocpage}}{}
\etocsetstyle{subsubsection}{}{}{\etocsavedsubsectiontocline{\numberline{}\etocname}{\etocpage}}{}
\etocsetstyle{paragraph}{}{}{\etocsavedsubsubsectiontocline{\numberline{}\etocname}{\etocpage}}{}
\etocsetstyle{subparagraph}{}{}{\etocsavedparagraphtocline{\numberline{}\etocname}{\etocpage}}{}


% Simple custom commands
% ======================
\makeatletter
\def\maxwidth#1{\ifdim\Gin@nat@width>#1 #1\else\Gin@nat@width\fi} 
\makeatother

\def\todo#1{\selectfont{\color{red}\texttt{\textbf{TODO:} #1}}}

% Sections 'n' such
% =================

\definecolor{chaptColor}{RGB}{0, 83, 161}

\def\chapt#1{
%
	% Default chapter behavior
	\begingroup\color{chaptColor}
	\chapter{#1}
	\endgroup
	
	% Label
	\label{chp:#1}
}

\definecolor{sectColor}{rgb}{0, 0.5, 0.0}
\def\sect#1{\textcolor{sectColor}{\section{#1}}}

\definecolor{subsectColor}{rgb}{0, 0.5, 0.5}
\def\subsect#1{\textcolor{subsectColor}{\subsection{#1}}\noindent}

\definecolor{subsubsectColor}{rgb}{0.747, 0.458, 0}
\def\subsubsect#1{\textcolor{subsubsectColor}{\subsubsection{#1}}}

% TL; DR section
% ==============
\newenvironment{tldr}{\begin{tcolorbox}[colback=gray!20!white,colframe=blue!75!black,title=TL; DR]}{\end{tcolorbox}\vspace*{12pt}}

% Custom math commands
% ====================
\DeclarePairedDelimiter\ceil{\lceil}{\rceil}
\DeclarePairedDelimiter\floor{\lfloor}{\rfloor}

% ======================
% \graphic{filename=...}
% ======================
% Keyword arguments
\makeatletter
\define@key{graphicKeys}{filename}{\def\graphic@filename{#1}}
\define@key{graphicKeys}{scale}{\def\graphic@scale{#1}}
\define@key{graphicKeys}{width}{\def\graphic@width{#1}}
\define@key{graphicKeys}{caption}{\def\graphic@caption{#1}}
\define@key{graphicKeys}{captionType}{\def\graphic@captionType{#1}}
\define@key{graphicKeys}{label}{\def\graphic@label{#1}}

% Kwargs
\DeclareDocumentCommand{\graphic}{m}{

	\begingroup
	
	% Set default kwargs
	\setkeys{graphicKeys}{filename={0}, #1}
	\setkeys{graphicKeys}{scale={0}, #1}
	\setkeys{graphicKeys}{width={0}, #1}
	\setkeys{graphicKeys}{caption={}, #1}
	\setkeys{graphicKeys}{captionType={figure}, #1}
	\setkeys{graphicKeys}{label={}, #1}
	
	% Assign width
	\let\graphicWidth\relax % let \mytmplen to \relax
	\newlength{\graphicWidth}
	\setlength{\graphicWidth}{\columnwidth}
	
	\if \graphic@scale 0
		
		\if \graphic@width 0
			
			\setlength{\graphicWidth}{0.5\columnwidth}
		
		\else
	
			\setlength{\graphicWidth}{\graphic@width}
			
		\fi

	\else
		
		\setlength{\graphicWidth}{\columnwidth * \graphic@scale}

	\fi	
	
	% Do figure
	\begin{minipage}{\columnwidth}
		\vspace*{12pt}
		\begin{center}
		
			\includegraphics[width = \graphicWidth]{\graphic@filename}
			
			\ifthenelse{\equal{\graphic@caption}{}}{}{
				\captionof{\graphic@captionType}{\graphic@caption}
			}
			
			\ifthenelse{\equal{\graphic@label}{}}{
				\label{fig: \graphic@filename}
			}{
				\label{\graphic@label}
			}
		\end{center}
		\vspace*{12pt}
	\end{minipage}
	
	\endgroup
}
\makeatother

% ==============
% Base Pair Page
% ==============
% Keyword arguments
\makeatletter
\define@key{bpPageKeys}{baseFilename}{\def\bpPage@baseFilename{#1}}
\define@key{bpPageKeys}{scale}{\def\bpPage@scale{#1}}
\define@key{bpPageKeys}{width}{\def\bpPage@width{#1}}
\define@key{bpPageKeys}{caption}{\def\bpPage@caption{#1}}
\define@key{bpPageKeys}{captionType}{\def\bpPage@captionType{#1}}
\define@key{bpPageKeys}{label}{\def\bpPage@label{#1}}

% Kwargs
\DeclareDocumentCommand{\BPPage}{m}{

	\begingroup
	
	% Set default kwargs
	\setkeys{bpPageKeys}{baseFilename={0}, #1}
	\setkeys{bpPageKeys}{caption={}, #1}
	\setkeys{bpPageKeys}{label={}, #1}
	
	\newpage

	\begin{center}
		\ul{\mbox{\bpPage@baseFilename{}: 1 Base Pair [min]}}
	\end{center}
	\vspace*{-36pt}
	\begin{minipage}[b]{0.5\textwidth}
		\graphic{filename=\bpPage@baseFilename-overview-1-table, width=\textwidth}
	\end{minipage}
	\begin{minipage}[b]{0.5\textwidth}
		\graphic{filename=\bpPage@baseFilename-overview-1-plot, width=\textwidth}
	\end{minipage}

	\begin{center}
		\ul{\mbox{\bpPage@baseFilename{}: 50 Base Pairs [average]}}
	\end{center}
	\vspace*{-36pt}
	\begin{minipage}[b]{0.5\textwidth}
		\graphic{filename=\bpPage@baseFilename-overview-50-table, width=\textwidth}
	\end{minipage}
	\begin{minipage}[b]{0.5\textwidth}
		\graphic{filename=\bpPage@baseFilename-overview-50-plot, width=\textwidth}
	\end{minipage}
	
	\begin{center}
		\ul{\mbox{\bpPage@baseFilename{}: 100 Base Pairs [max]}}
	\end{center}
	\vspace*{-36pt}
	\begin{minipage}[b]{0.5\textwidth}
		\graphic{filename=\bpPage@baseFilename-overview-100-table, width=\textwidth}
	\end{minipage}
	\begin{minipage}[b]{0.5\textwidth}
		\graphic{filename=\bpPage@baseFilename-overview-100-plot, width=\textwidth}
	\end{minipage}

	\vspace*{-24pt}
	\captionof{figure}{\bpPage@caption}\label{\bpPage@label}	
	
	\endgroup
}
\makeatother



\begin{document}
  		\begin{document}
		\renewcommand{\preloaded}{false}
	\fi
\fi
\makeatother
\trysetmain{}

% Additional usage: \ifthenelse{\equal{\preloaded}{true}}{ ... }{ ... }
% Note: Cannot do \trysetmain{} here because the \currfilename will always be "standalone.tex"

% ===========================================================
% Outlet cell function. Usage inside OutletTable environment:
%	\OutletCell{title= ..., 
%				beforeParams={ ... }, 
%				beforeNote={ Optional! },
%				afterParams={\code{const float OriginalYield},\\
%							 \code{const float ReturnedYield}
%							},
%				afterNote={ Also optional! }
%				}\\
% ===========================================================

% Keyword arguments
\makeatletter
\define@key{outletKeys}{title}{\def\outletKeys@title{#1}}
\define@key{outletKeys}{beforeParams}{\def\outletKeys@beforeParams{#1}}
\define@key{outletKeys}{afterParams}{\def\outletKeys@afterParams{#1}}
\define@key{outletKeys}{beforeNote}{\def\outletKeys@beforeNote{#1}}
\define@key{outletKeys}{afterNote}{\def\outletKeys@afterNote{#1}}

% Kwargs
\DeclareDocumentCommand{\OutletCell}{m}{

	\begingroup
	
	% Set default kwargs
	\setkeys{outletKeys}{title={0}, #1}
	\setkeys{outletKeys}{beforeParams={0}, #1}
	\setkeys{outletKeys}{afterParams={0}, #1}
	\setkeys{outletKeys}{beforeNote={}, #1}
	\setkeys{outletKeys}{afterNote={}, #1}

	% Title
	\begin{tblr}{
		colspec={Q[l, wd=\linewidth]},
		leftsep = 0pt,
		rightsep = 0pt,
		hline{1} = {2pt},
		hline{2} = {1.5pt},
	}
	\color{ICBlue}\textbf{\hspace*{0.5em}\outletKeys@title}
	\end{tblr}	
	
	% Before
	\begin{tblr}{
		colspec={Q[l, wd=0.2\linewidth] Q[l, wd=0.8\linewidth]},
		leftsep = 0pt,
		rightsep = 0pt,
	}
		\hspace*{1em}$\blacktriangleright$ Before	& {	\outletKeys@beforeParams }
	\end{tblr}
	
	% Before note
	\ifthenelse{\equal{\outletKeys@beforeNote}{}}{}{%
		\begin{tblr}{
			colspec={Q[l, wd=0.2\linewidth] Q[l, wd=0.8\linewidth]},
			leftsep = 0pt,
			rightsep = 0pt,
		}
			\hspace*{2em} \textit{Note:} & \outletKeys@beforeNote
		\end{tblr}
	}
	
	% After
	\begin{tblr}{
		colspec={Q[l, wd=0.2\linewidth] Q[l, wd=0.8\linewidth]},
		leftsep = 0pt,
		rightsep = 0pt,
		hline{1} = 1pt,
	}
		\hspace*{1em}$\blacktriangleright$ After	& {	\outletKeys@afterParams }
	\end{tblr}
	
	% After note
	\ifthenelse{\equal{\outletKeys@afterNote}{}}{}{%
		\begin{tblr}{
			colspec={Q[l, wd=0.2\linewidth] Q[l, wd=0.8\linewidth]},
			leftsep = 0pt,
			rightsep = 0pt,
		}
			\hspace*{2em} \textit{Note:} & \outletKeys@afterNote
		\end{tblr}
	}
	
	\endgroup
}
\makeatother

% ================================
% Outlet table environment. Usage:
%	\begin{OutletTable}{Title}
%		...
%		[Use \OutletCell}
%		...
%	\end{OutletTable}
% ================================
\newenvironment{OutletTable}[1]{%
\begin{longtblr}[
	caption = {\code{Outlet}s for \code{#1}},
	label = {delegate-arrays-levelcomponent},
]{
	cells = {gray!15},
	leftsep = 0pt,
	rightsep = 0pt,
	colspec={|Q[c, wd=\linewidth]|},
	hline{2-Z}={2pt},
	%rowsep=-0.5em,
	rows = {abovesep=-12pt},
}
}{%
\end{longtblr}
}



%--------------------------

\begin{OutletTable}{AffinitiesComponent}

	\OutletCell{title=GetUnspentPoints, 
				beforeParams={\code{const uint8 OriginalPoints},\\
							\code{uint8\& ReturnedPoints}},
				afterParams={\code{const uint8 OriginalPoints},\\
							\code{const uint8 ReturnedPoints}}
				}
				\\
				
	\OutletCell{title=SetUnspentPoints, 
				beforeParams={\code{const uint8 OriginalPoints},\\
							\code{const uint8 InputPoints},\\
							\code{uint8\& SetPoints}},
				afterParams={\code{const uint8 OriginalPoints},\\
							\code{const uint8 InputPoints},\\
							\code{const uint8 SetPoints}}
				}
				\\
	
\end{OutletTable}

%--------------------------

\gap{}

\begin{OutletTable}{EffectComponent}

	\OutletCell{title=GetStacks, 
				beforeParams={\code{const uint16 OGStacks},\\
							\code{int32\& ReturnedStacks}},
				afterParams={\code{const uint16 OGStacks},\\
							\code{const int32 ReturnedStacks}},
				}
				\\
				
	\OutletCell{title=OnAddEffect, 
				beforeParams={\code{const EffectComponent* EffectToAdd}
							},
				afterParams={\code{const EffectComponent* AddedEffect}
							},
				}
				\\
				
	\OutletCell{title=OnRemoveEffect, 
				beforeParams={\code{const EffectComponent* EffectToRemove}
							},
				afterParams={\code{const EffectComponent* RemovedEffect}
							},
				}
				\\
	
\end{OutletTable}

%--------------------------

\gap{}

\begin{OutletTable}{LevelComponent}

	\OutletCell{title=GetBaseExpYield, 
				beforeParams={\code{const float OriginalYield},\\
							\code{float\& ReturnedYield}},
				afterParams={\code{const float OriginalYield},\\
							\code{const float ReturnedYield}}
				}
				\\
				
	% ----------------------------------------
				
	\OutletCell{title=GetCXP, 
				beforeParams={\code{const uint32 OriginalCXP},\\ 
							\code{int32\& ReturnedCXP}},
				beforeNote={\code{ReturnedCXP} is \code{int32\&} instead of \code{uint32\&} for Blueprint compatability.},
				afterParams={\code{const uint32 OriginalCXP}\\ 
							\code{const int32 ReturnedCXP}},
				afterNote={\code{ReturnedCXP} is \code{const int32} instead of \code{const uint32} for Blueprint compatability.}
				}	
				\\			

	% ----------------------------------------				
				
	\OutletCell{title=GetExpYield, 
				beforeParams={\code{const float OriginalYield},\\ 
							\code{float\& ReturnedYield},\\
							\code{const uint16 DefeatedLevel},\\
							\code{const uint16 VictoriousLevel}},
				beforeNote={``Defeated'' and ``Victorious'' levels are provided for flexibility (e.g., in case you want to yield exp differently based on level difference, although technically you could always back-calculate the level difference based on the equation and \code{OriginalYield}).},
				afterParams={\code{const float OriginalYied},\\ 
							\code{const float ReturnedYield},\\
							\code{const uint16 DefeatedLevel},\\
							\code{const uint16 VictoriousLevel}},
				afterNote={``Defeated'' and ``Victorious'' levels are provided for symmetry with respect to the \code{Before} delegate (since \code{ReturnedValue} is already calculated, I can't think of why you would need them, but you never know!).}
				}
				\\
			
	% ----------------------------------------			
				
	\OutletCell{title=GetMaxLevel, 
				beforeParams={\code{const uint16 DefaultMax},\\ 
							\code{int32\& AttemptedMax}},
				beforeNote={\code{DefaultMax} is defined in the code. It should normally be 100, but may change for certain subclasses (e.g., a \code{BossLevelComponent} may have a max of 200 instead). \\ 
				\code{AttemptedMax} is \code{int32\&} instead of \code{uint16\&} for Blueprint compatability.
				},
				afterParams={\code{const uint16 DefaultMax}\\ 
							\code{const int32 ReturnedMax}},
				}
				\\

	% ----------------------------------------

	\OutletCell{title=GetMinLevel, 
				beforeParams={\code{const uint16 DefaultMin},\\ 
							\code{int32\& AttemptedMin}},
				beforeNote={\code{DefaultMin} is defined in the code. It should normally be 1, but may change for certain subclasses (e.g., a \code{EggLevelComponent} may have a min of 0 instead for whatever reason).\\
				\code{AttemptedMin} is \code{int32\&} instead of \code{uint16\&} for Blueprint compatability.},
				afterParams={\code{const uint16 DefaultMin}\\ 
							\code{const int32 ReturnedMin}},
				afterNote={\code{ReturnedCXP} is \code{const int32} instead of \code{const uint32} for Blueprint compatability.}
				}
				\\
		
	% ----------------------------------------		
				
	\OutletCell{title=GetBaseExpYield, 
				beforeParams={\code{const float OriginalYield},\\
							\code{float\& ReturnedYield}},
				afterParams={\code{const float OriginalYield},\\
							\code{const float ReturnedYield}}
				}
				\\
		
	% ----------------------------------------		
				
	\OutletCell{title=SetBaseExpYield, 
				beforeParams={\code{const float OldYield},\\
							\code{const float InputYield},\\ 
							\code{float\& AttemptedYield}},
				afterParams={\code{const float OldYield}\\
							\code{const float InputYield},\\ 
							\code{const float NewYield}},
				afterNote={$\triangleright$ \code{OldYield} is the yield prior to calling \code{SetBaseExpYield},\\
							$\triangleright$ \code{InputYield} is the original, unmodified input to \code{SetBaseExpYield},\\
							$\triangleright$ \code{AttemptedYield} is the modified value that will be used to set the base exp yield.
							}
				}
				\\
		
	% ----------------------------------------		
						
	\OutletCell{title=SetCXP, 
				beforeParams={\code{const uint32 OldCXP},\\ 
							\code{const int32 InputCXP},\\
							\code{int32\& AttemptedCXP}},
				beforeNote={\code{AttemptedCXP} is \code{int32\&} instead of \code{uint32\&} for Blueprint compatability.},
				afterParams={\code{const uint32 OldCXP}\\ 
							\code{const int32 InputCXP},\\
							\code{const uint32 NewCXP}},
				afterNote={\code{StatsComponent} subscribes to this in order to change stats on level change.\\
							$\triangleright$ \code{OldCXP} is the cumulative experience points prior to calling \code{SetCXP},\\
							$\triangleright$ \code{InputCXP} is the original, unmodified input to \code{SetCXP},\\
							$\triangleright$ \code{AttemptedCXP} is the modified value that will be used to set the cumulative experience points.		
					}
				}
				\\
		
\end{OutletTable}

%--------------------------

\gap{}

\begin{OutletTable}{StatsComponent}

	% ----------------------------------------		
						
	\OutletCell{title=ApplyDamage, 
				beforeParams={
							\code{float\& BasePower},\\
							\code{float\& CritMultiplier},\\
							\code{float\& RandFluct},\\
							\code{float\& Stab},\\
							\code{float\& TypeAdvantage},\\
							\code{UCombatStatsComponent* Attacker},\\
							\code{UCombatStatsComponent* OwningStats}
							},
				beforeNote={Other quantities, such as the Attacker's attacking Stat, may be calculated from the given quantities.},
				afterParams={
							\code{const float BasePower},\\
							\code{const float CritMultiplier},\\
							\code{const float RandFluct},\\
							\code{const float\& Stab},\\
							\code{const float\& TypeAdvantage},\\
							\code{UCombatStatsComponent* Attacker},\\
							\code{UCombatStatsComponent* OwningStats}
							}
				}
				\\

	% ----------------------------------------		
						
	\OutletCell{title=CalculateDamage, 
				beforeParams={
							\code{float\& BasePower},\\
							\code{float\& CritMultiplier},\\
							\code{float\& RandFluct},\\
							\code{float\& Stab},\\
							\code{float\& TypeAdvantage},\\
							\code{UCombatStatsComponent* Attacker},\\
							\code{UCombatStatsComponent* OwningStats}
							},
				afterParams={
							\code{const float BasePower},\\
							\code{const float CritMultiplier},\\
							\code{const float RandFluct},\\
							\code{const float\& Stab},\\
							\code{const float\& TypeAdvantage},\\
							\code{UCombatStatsComponent* Attacker},\\
							\code{UCombatStatsComponent* OwningStats}
							},
				afterNote={The difference between calculating damage and applying damage is theoretical. For example, low-level AI might use \code{CalculateDamage} to make decisions. On the other hand, applying the damage might invoke some kind of reaction, like raising Physical Attack if hit by a move it's weak to.}
				}
				\\

	% ----------------------------------------		
						
	\OutletCell{title=GetCritMult, 
				beforeParams={\code{float\& BaseMultiplier},\\
							\code{float\& CritBonus},\\
							\code{UCombatStatsComponent* OwningStats}
							},
				beforeNote={Total crit bonus is \code{BaseMultiplier} + \code{CritBonus}; for example, 1.5 + 0.2.},
				afterParams={\code{const float BaseMultiplier},\\
							\code{const float CritBonus},\\
							\code{UCombatStatsComponent* OwningStats}
							}
				}
				\\

	% ----------------------------------------		
						
	\OutletCell{title=ModifyStat, 
				beforeParams={\code{const EStatEnum TargetStat},\\
							\code{const EStatValueType ValueType},\\
							\code{const EModificationMode Mode},\\
							\code{const float OriginalValue},\\
							\code{float\& AttemptedValue}
							},
				afterParams={\code{const EStatEnum TargetStat},\\
							\code{const EStatValueType ValueType},\\
							\code{const EModificationMode Mode},\\
							\code{const float OriginalValue},\\
							\code{const float NewValue}
							},
				afterNote={All ``ModifyStat'' functions from \code{StatsComponent} (such as \code{ModifyStatsUniformly} or \code{RandomizeStats}) go through \code{ModifyStatInternal}, which calls this \code{Outlet}.}
				}
				\\

	% ----------------------------------------		
						
	\OutletCell{title=RandomizeStats, 
				beforeParams={\code{const EStatEnum TargetStat},\\
							\code{const FStatRandParams OriginalParams},\\
							\code{FStatRandParams\&  ParamsToBeUsed}},
				afterParams={\code{const EStatEnum TargetStat},\\
							\code{const FStatRandParams OriginalParams},\\
							\code{const FStatRandParams UsedParams}},
				afterNote={The \code{EStatEnum} is not the acutal \code{FStat}. To get the \code{FStat} (such as \code{FHealth}), use \code{StatsComponent::GetStat(EStatEnum)}}
				}
				\\
		
	% ----------------------------------------		
						
	\OutletCell{title=RecalculateStats, 
				beforeParams={\code{const EStatEnum TargetStat},\\
							\code{const bool bResetCurrent},\\
							\code{const float OriginalCurrent},\\
							\code{const float OriginalPermanent}},
				afterParams={\code{const EStatEnum TargetStat},\\
							\code{const bool bResetCurrent},\\
							\code{const float OriginalCurrent},\\
							\code{const float OriginalPermanent}},
				}
				\\
			
		
	
\end{OutletTable}


\postamble{}

%====================================================

\sect{Making Your Own \code{Outlet}}

As an example, let's use \code{GetBaseExpYield}. (You can imagine that this is an important \code{Outlet} for tweaking levelling curves.) Here's what to do:\\

\begin{enumerate}
	\item{\textbf{Plan ahead.} I would sincerely recommend you writing down what parameters your \code{Outlet} \code{Before} and \code{After} delegates take on paper. We go to a few files and it's easy to be inconsistent.}
	\item{\textbf{Go to the right directory.} We want to place the \code{Outlet} inside of \code{ULevelComponent}, so we'll start with that directory. If yours doesn't contain an ``Outlets'' directory, create one and place your \code{Outlet}(s) there.}
	\item{\textbf{Copy + paste file.} The easiest way is to copy + paste pre-existing \code{Outlet}s. In this example, we'll copy + paste \code{SetCXPOutlet.h} and name the new file \code{GetBaseExpYield.h}. \\
	\begin{center}
		\includegraphics[scale=\ScreenshotScale]{create-outlet-rename}
	\end{center}
	\noindent \textit{Note: this includes both \code{BeforeGetBaseExpYield} and \code{AfterGetBaseExpYield} functionality. You don't have to make two different files!}
	}
	\item{\textbf{Replace old name.} Open the new file and you'll still see the base name ``SetCXP'' everywhere. The easiest way is to do a find+replace ``SetCXP'' $\rightarrow$ ``GetBaseExpYield''. This replaces everything from the \code{.generated} include to the delegate signatures. If you're curious, you can look more in-depth and replace instances one-by-one.}
	\item{\textbf{Declare delegate signatures.} In this case, we want the \code{Before} delegate signature to take two arguments: the original, unmodified yield and the one that will be returned from the \code{GetBaseExpYield} function.\\
	\begin{center}
		\includegraphics[scale=\ScreenshotScale]{create-outlet-signature}
	\end{center}
	You should also set the \code{After} signature in the same manner. \textit{Note: yours might use more than two parameters or different parameter types. Modify accordingly.}
	}
	\item{\textbf{Module API.} Make sure your module API is correct. If not, you'll get mysterious errors about your dll.
	\begin{center}
		\includegraphics[scale=\ScreenshotScale]{create-outlet-api}
	\end{center}
	}
	\item{\textbf{Declare \code{Outlet} functions.} In order to be able to call \code{ExecuteBefore} on your \code{Outlet}, you need to tell it a few things. The figure below displays a few things in red you should look at:\\
	\begin{center}
		\includegraphics[scale=\ScreenshotScale]{create-outlet-functions}
	\end{center}
		\begin{itemize}
			\item{Whether it's a \code{Before} or \code{After} type \code{Outlet}. This affects execution based on priority:
			\begin{tcolorbox}[colback=gray!20!white,colframe=blue!75!black,title=Priorities]
			
				The lower the priority, the farther away it is from execution. If two priorities are tied, the older effect is executed first. Order is set externally by \code{UEffectsComponent} \todo{fact check this}. Order:\\
				\begin{itemize}
					\item{Intrinsic \code{Before} delegates (no \code{UEffect} affiliated)}
					\item{\code{Before} delegates:
						\begin{itemize}
							\item{Priority 1}
							\item{Priority 2.a (older)}
							\item{Priority 2.b (newer)}
							\item{...}
						\end{itemize}
						}
					\item{[Function executes]}
					\item{\code{After} delegates:
						\begin{itemize}
							\item{...}
							\item{Priority 2.b (newer)}
							\item{Priority 2.a (older)}
							\item{Priority 1}
						\end{itemize}
						}
					\item{Intrinsic \code{After} delegates (have the final say)}
				\end{itemize}
				
			As an example, consider two delegates: one that says you can't take damage no matter what (call the \code{UBuff} ``Invincible'') and another that says damage against you can't be avoided no matter what (call the \code{UDebuff} ``Weakened''). What happens when the target takes damage? Well, it depends on priority:
			\begin{itemize}
				\item{They're probably subscribed to the \code{Before} delegate in \code{UStatsComponent} called \code{ModifyStatOutlet} with the target \code{FStat} being \code{Health}.}
				\item{Note that they're both \code{Before} delegates.}
				\item{Let's say Invincible has Priority 100 and Weakened has Priority 150. The result is the target takes damage because:
					\begin{enumerate}
						\item[1)]{Invincible first sets the damage to zero.}
						\item[2)]{Weakened then sets the damage to no less than its original value.}
					\end{enumerate}
				}
				\item{If Weakened has lower Priority, the result is flipped and the target takes no damage.}
			\end{itemize}
	 
	 \end{tcolorbox}\vspace*{12pt}
			}
			\item{The parameters you defined in the delegate's signature. I know, I know---anytime you repeat code, you're probably doing something wrong. The biggest issue here is the UHT. The main (but not only) issue is that you can't have \code{UPROPERTY}s inside macros or the property won't register. If you have a better way of automating this, \textit{tell me!}}
			\item{Don't forget the \code{After} variant's delegates, which should probably be \code{const}.}
		\end{itemize}
	}
	\item{\textbf{Check number of parameters.} I make a point of this because I find it's my most common error. Make sure your declared signature \textit{and} declared \code{Outlet} function macros have the correct number of params (two in our case). Explicitly, you might need to use \code{DECLARE\_DYNAMIC\_DELEGATE\_FourParams(...)}.}
	\item{\textbf{Declare \code{UPROPERTY}.} Inside the \code{UEffectableComponent} (in this example, \code{ULevelComponent}), declare the \code{Outlet} as a variable. Note that it's custom to have this \code{UPROPERTY} as public and in the ``Outlets'' category. It's also a good idea to comment the \code{UPROPERTY} with the parameters.
	\begin{center}
		\includegraphics[scale=\ScreenshotScale]{create-outlet-uproperty}
	\end{center}
	
	\textit{Note:} I use Rider, so it imports \code{\#include}s automatically. Make sure yours does, too.
	}
	\item{\textbf{Implement.} Now it's time to place your \code{Outlet} in the appropriate place(s). For our example, it's pretty simple: place it inside of \code{GetBaseExpYield} in \code{ULevelComponent}'s \code{.cpp} file.
	\begin{center}
		\includegraphics[scale=\ScreenshotScale]{create-outlet-cpp}
	\end{center}
	Note that you might have to do things like cache original values.
	}
	\item{\textbf{A note on complementary delegates.} If you create a \code{Before} \code{Outlet}, you should also create an \code{After} \code{Outlet}. The biggest difference might be the delegate signature (e.g., reference ``\code{\&}'' to \code{const}).
			
	An example where this would be necessary is an animation delegate. You only want to fire a ``bonus exp'' animation \textit{after} the amount of exp has been determined, checked, and is now constant.
	
	In some cases, it may not be necessary to have both \code{Before} and \code{After} delegates in a function. If you want only one delegate type, or three, or ten, the system is flexible enough to handle it. However, it's recommended to K.I.S.S.
	}
\end{enumerate}

\sect{Making Your Own Effects}

Suppose you want to make your own effect from scratch. \todo{todo}

\postamble{}